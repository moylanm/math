\documentclass[11pt]{article}

\usepackage{color, enumitem, graphicx, amsmath, amsthm, amssymb}
\usepackage[margin=.5in]{geometry}
\usepackage[T1]{fontenc} % Use 8-bit encoding that has 256 glyphs

\usepackage[english]{babel} % English language/hyphenation

\usepackage{sectsty} % Allows customizing section commands
\allsectionsfont{\normalfont\scshape} % Make all sections centered, the default font and small caps

\usepackage{fancyhdr} % Custom headers and footers
\pagestyle{fancyplain} % Makes all pages in the document conform to the custom headers and footers
\fancyhead{} % No page header - if you want one, create it in the same way as the footers below
\fancyfoot[L]{} % Empty left footer
\fancyfoot[C]{} % Empty center footer
\fancyfoot[R]{\thepage} % Page numbering for right footer
\renewcommand{\headrulewidth}{0pt} % Remove header underlines
\renewcommand{\footrulewidth}{0pt} % Remove footer underlines
\setlength{\headheight}{13.6pt} % Customize the height of the header

\setlength\parindent{2em}

\graphicspath{ {./} }

\renewcommand\qedsymbol{$\blacksquare$}

%-------------------------------------------------------------------------------
%	TITLE SECTION
%-------------------------------------------------------------------------------

%\newcommand{\horrule}[1]{\rule{\linewidth}{#1}} % Create horizontal rule command with 1 argument of height

\title{	
	\normalfont \normalsize 
	\textsc{Discrete Mathematics} \\
	%\horrule{0.5pt} \\[0.4cm] % Thin top horizontal rule
	\huge Counting \\
	%\horrule{2pt} \\[0.5cm] % Thick bottom horizontal rule
}

\author{Myles Moylan} % Your name

\date{} % Today's date or a custom date


%-------------------------------------------------------------------------------
%	WORK SECTION
%-------------------------------------------------------------------------------

\begin{document}
	
\maketitle

\section*{\textbf{6.1 The Basics of Counting}}
\begin{enumerate}[label=\textbf{\arabic*.}]
	\item There are 18 mathematics majors and 325 computer science majors at a college.
	
	\begin{enumerate}[label=\textbf{\alph*)}]
		\item In how many ways can two representatives be picked so that one is a mathematics major and the other is a computer science major?
		
		By the product rule, $18 \cdot 325 = 5850$ way to pick the two representatives.
		
		\item In how many ways can one representative be picked who is either a mathematics major or a computer science major.
		
		By the sum rule, $18 + 325 = 343$ ways to pick the representative.
	\end{enumerate}

	\item An office building contains 27 floors and had 37 offices on each floor. How many offices are in the building?
	
	$27 \cdot 37 = 999$ offices in the building.
	
	\item A multiple-choice test contains 10 questions. There are four possible answers for each question.
	
	\begin{enumerate}[label=\textbf{\alph*)}]
		\item In how many ways can a student answer the questions on the test if the student answers every question?
		
		$10 \cdot 4 = 40$ ways to answer the questions on the test.
		
		\item In how many ways can a student answer the questions on the test if the student can leave answers blank?
		
		$10 \cdot (4 + 1) = 50$ ways to answer the questions if leaving blank answers is an option.
	\end{enumerate}

	\item A particular brand of shirt comes in 12 colors, has a male and a female version, and comes in three sizes for each sex. How many different types of this shirt are made?
	
	$(3 \cdot 12) + (3 \cdot 12) = 72$ different types of shirt.
	
	\item Six different airlines fly from New York to Denver and seven fly from Denver to San Francisco. How may different pairs of airlines can you choose on which to book a trip from New York to San Francisco via Denver, when you pick an airline for the flight to Denver and an airline for the continuation flight to San Francisco?
	
	$6 \cdot 7 = 42$ differently possible flight plans.
	
	\item There are four major auto routes from Boston to Detroit and six from Detroit to Las Angeles. How many major auto routes are there from Boston to Los Angeles via Detroit?
	
	$4 \cdot 6 = 24$ major auto routes from Boston To LA via Detroit.
	
	\item How many different three-letter initials can people have?
	
	$26 \cdot 26 \cdot 26 = 26^3 = 17576$ possible three-letter initials.
	
	\item How many different three-letter initials with none of the letters repeated can people have?
	
	$26 \cdot 25 \cdot 24 = 15600$ different possible three-letter initials without repeating letters.
	
	\item How many different three-letter initials are there that begin with an A?
	
	$1 \cdot 26 \cdot 26 = 26^2 = 676$ possible three-letter initials beginning with A.
	
	\item How many bit strings are there of length eight?
	
	$2 \cdot 2 \cdot 2 \cdot 2 \cdot 2 \cdot 2 \cdot 2 \cdot 2 = 2^8 = 256$ possible bit strings of length eight.
	
	\item How many bit strings of length ten both begin and end with a 1?
	
	$1 \cdot 2 \cdot 2 \cdot 2 \cdot 2 \cdot 2 \cdot 2 \cdot 2 \cdot 2 \cdot 1 = 2^8 = 256$ possible strings of length ten that begin and end with a 1.
	
	\item How many bit strings are there of length six or less, not counting the empty string?
	
	$2^6 + 2^5 + 2^4 + 2^3 + 2^2 + 2^1 = 126$ different bit strings of length six or less, not counting the empty string.
	
	\item How many bit strings of length not exceeding $n$, where $n$ is a positive integer, consist entirely of 1s, not counting the empty string?
	
	Since the string is given to consist entirely of 1's, there is nothing to choose except the length. Since there are $n + 1$ possible lengths not exceeding $n$ (if we include the empty string, of length 0), the answer is simply $n + 1$. Note that the empty string consists---vacuously---entirely of 1's.
	
	\item How many bit strings of length $n$, where $n$ is a positive integers, start and end with 1s?
	
	$2^{n - 2}$ different bit strings of length $n$ which start and end with 1s.
	
	\item How many strings are there of lowercase letters of length four or less, not counting the empty string?
	
	By the sum rule we can count the number of strings of length 4 or less by counting the number of strings of length $i$, for $0 \leq i \leq 4$, and then adding the results.
	
	$$\sum_{i=0}^{4} 26^i = 1 + 26 + 676 + 17576 + 456976 = 475255$$
	
	\item How many strings are there of four lowercase letters that have the letter $x$ in them?
	
	Here, for example, we use the product rule to count the number of possible strings: $26 \cdot 26 \cdot 26 \cdot 1$, where the 1 is the location of the $x$. And since the $x$ can be in any of four places we multiply the product by 4.
	
	$26^3 \cdot 4 = 70304$ possible strings of four lowercase letters that have the letter $x$ in them.
	
	\item How many strings of five ASCII characters contain the character @ ("at" sign) at least once? [\emph{Note:} There are 128 different ASCII characters.]
	
	An easy way to count this is to find the total number of ASCII strings of length five and then subtract off the number of such strings that do not contain the @ character. Since there are 128 characters to choose from in each location in the string, the answer is $128^5 - 127^5 = 34359738368 - 33038369407 = 1321368961$.
	
	\item How many 6-element RNA sequences
	
	Recall that an RNA sequence is a sequence of letters, each of which is one of A, C, G, or U. Thus by the product rule there are $4^6$ RNA sequences of length six if we impose no restrictions.
	
	\begin{enumerate}[label=\textbf{\alph*)}]
		\item do not contain U?
		
		If U is excluded, then each position can be chosen from among three letters, rather than four. Therefore the answer is $3^6 = 729$.
		
		\item end with GU?
		
		If the last two letters are specified, then we get to choose only four letters, rather than six, so the answer is $4^4 = 256$
		
		\item start with C?
		
		If the first letter is specified, then we get to choose only five letters, rather than six, so the answer is $4^5 = 1024$.
		
		\item contain only A or U?
		
		If only A or U is allowed in each position, then there are just two choices at each of six stages, so the answer is $2^6 = 64$.
	\end{enumerate}

	\item How many positive integers between 50 and 100
	
	Because neither 100 nor 50 is divisible by either 7 or 11, whether the ranges are meant to be inclusive or exclusive of their endpoints is moot.
	
	\begin{enumerate}[label=\textbf{\alph*)}]
		\item are divisible by 7? Which integers are these?
		
		There are $\lfloor 100 / 7 \rfloor = 14$ integers less than 100 that are divisible by 7, and $\lfloor 50 / 7 \rfloor = 7$ of them are less than 50 as well. This leaves $14 - 7 = 7$ numbers between 50 and 100 that are divisible by 7. They are 56, 63, 70, 77, 84, 91, and 98.
		
		\item are divisible by 11? Which integers are these?
		
		There are $\lfloor 100 / 11 \rfloor = 9$ integers less than 100 that are divisible by 11, and $\lfloor 50 / 11 \rfloor = 4$ of them are less than 50 as well. This leaves $9 - 4 = 5$ numbers between 50 and 100 that are divisible by 11. They are 55, 66, 77, 88, and 99.
		
		\item are divisible by both 7 and 11? Which integers are these?
		
		A number is divisible by both 7 and 11 if and only if it is divisible by their least common multiple, which is 77. There is only one such number between 50 and 100, namely 77.
	\end{enumerate}

	\item How many strings of three decimal digits
	
	This problem involves 1000 possible strings, since there is a choice of 10 digits for each of the three positions in the string.
	
	\begin{enumerate}[label=\textbf{\alph*)}]
		\item do not contain the same digit three times?
		
		This is most easily done by subtracting from the total number of strings the number of strings that violate the condition. There are 10 strings that consist of the same digit three times (000, 111, $\ldots$, 999). Therefore there are $1000 - 10 = 990$ strings that do not.
		
		\item begin with an odd digit?
		
		If we must begin our string with an odd digit, then we have only 5 choices for this digit. We still have 10 choices for the remaining digits. Therefore there are $5 \cdot 10 \cdot 10 = 500$ such strings.
		
		\item have exactly two digits that are 4s?
		
		Here we need to choose the position of the digits that is not a 4 (3 ways) and choose that digit (9 ways). Therefore there are $3 \cdot 9 = 27$ such strings.
	\end{enumerate}
\end{enumerate}

\section*{\textbf{6.2 The Pigeonhole Principle}}
\begin{enumerate}[label=\textbf{\arabic*.}]
	\item Show that in any set of six classes, each meeting regularly once a week on a particular day of the week, there must be two that meet on the same day, assuming that no classes are held on weekends.
	
	There are six classes: these are the pigeons. There are five days on which classes may meet (Monday through Friday): these are the pigeonholes. Each class must meet on a day (each pigeon must occupy a pigeonhole). By the pigeonhole principle at least one day must contain at least two classes.
	
	\item A drawer contains a dozen brown socks and a dozen black socks, all unmatched. A man takes socks out at random in the dark.
	
	\begin{enumerate}[label=\textbf{\alph*)}]
		\item How many socks must he take out to be sure that he has at least two socks of the same color?
		
		There are two colors: these are the pigeonholes. We want to know the least number of pigeons needed to insure that at least one of the pigeonholes contains two pigeons. By the pigeonhole principle the answer is 3. If three socks are taken from the drawer, at least two much have the same color. On the other hand two sock are not enough, because one might be brown and the other black.
		
		\item How many socks must he take out to be sure that he has at least two black socks?
		
		He needs to take out 14 socks in order to insure at least two black socks. If he does so, then at most 12 of them are brown, so at least two are black. On the other hand, if he removes 13 or fewer socks, then 12 of them could be brown, and he might not get his pair of black socks.
	\end{enumerate}

	\item Show that if there are 30 students in a class, then at least two have last names that begin with the same letter.
	
	There are 26 letters in the English alphabet (these are the pigeons) and 30 students in the class (these are the pigeonholes). By the pigeonhole principle, actually, there only need be 27 students in the class for there to be two students with last names the begin with the same letter.
	
	\item Undergraduate students at a college belong to one of four groups depending on the year in which they are expected to graduate. Each student must choose one of 21 different majors. How many students are needed to assure that there are two students expected to graduate in the same year who have the same major?
	
	In this question, the students are pigeons. The boxes or pigeonholes are less clear: since the question is asking about students expected to graduate in the same year with the same major, we think of a year-major pair as being a box. More formally, if $Y$ is the set of 4 years and $M$ is the set of 21 different majors, then $Y \times M$ is the set of boxes. This means that there are $4 \times 21 = 84$ boxes, and thus $84 + 1 = 85$ students that are needed to ensure that there are two in the same box.
	
	\item Show that among any group of five (not necessarily consecutive) integers, there are two with the same remainder when divided by 4.
	
	There are four possible remainders when an integer is divided by 4 (these are the pigeonholes here): 0, 1, 2, or 3. Therefore, by the pigeonhole principle at least two of the five given remainders (these are the pigeons) must be the same.
	
	\item Let $n$ be a positive integer. Show that in any set of $n$ consecutive integers there is exactly one divisible by $n$.
	
	Let the $n$ consecutive integers be denoted $x + 1, x + 2, \ldots, x + n$, where $x$ is some integer. We want to show that exactly one of these is divisible by $n$. There are $n$ possible remainders when an integer is divided by $n$, namely $0, 1, 2, \ldots, n - 1$. There are two possibilities for the remainders of our collection of $n$ numbers: either they cover all the possible remainders (in which case exactly one of our numbers has a remainder of 0 and is therefore divisible by $n$), or they do not. If they do not, then by the pigeonhole principle, since there are then fewer than $n$ pigeonholes (remainders) for $n$ pigeons (the numbers in our collection), at least one remainder must occur twice. In other words, it must be the case that $x + i$ and $x + j$ have the same remainder when divided by $n$ for some pair of numbers $i$ and $j$ with $0 < i < j \leq n$. Since $x + i$ and $x + j$ have the same remainder when divided by $n$, if we subtract $x + i$ from $x + j$, then we will get a number divisible by $n$. This means that $j - i$ is divisible by $n$. But this is impossible, since $j - i$ is a positive integer strictly less than $n$. Therefore the first possibility must hold, that exactly one of the numbers in our collection is divisible by $n$.
	
	\item What is the minimum number of students, each of whom come from one of the 50 states, who must be enrolled in a university to guarantee that there are at least 100 who come from the same state?
	
	The generalized pigeonhole principle applies here. The pigeons are the students, and the pigeonholes are the states, 50 in number. By the generalized pigeonhole principle if we want there to be at least 100 pigeons in at least one of the pigeonholes, then we need to have a total of $N$ pigeons so that $\lceil N / 50 \rceil \geq 100$. This will be the case as long as $N \geq 99 \cdot 50 + 1 = 4951$. Therefore we need at least 4951 students to guarantee that at least 100 come from a single state.
	
	\item Suppose that every student in a discrete mathematics class of 25 students is a freshman, a sophomore, or a junior.
	
	\begin{enumerate}[label=\textbf{\alph*)}]
		\item Show that there are at least nine freshmen, at least nine sophomores, or at least nine juniors in the class.
		
		If this statement were not true, then there would be at most 8 from each class standing, for a total of at most 24 students. This contradicts the fact that there are 25 students in the class.
		
		\item Show that there are either at least three freshmen, at least 19 sophomores, or at least five juniors in the class.
		
		If this statement were not true, then there would be at most 2 freshmen, at most 18 sophomores, and at most 4 juniors, for a total of at most 24 students. This contradicts the fact that there are 25 students in the class.
	\end{enumerate}

	\item Construct a sequence of 16 positive integers that has no increasing or decreasing subsequence of five terms.
	
	One way to do this is to have the sequence contain four groups of four numbers each, so that the numbers within each group are decreasing, and so that the numbers between groups are increasing. For example, we could take the sequence to be 4, 3, 2, 1; 8, 7, 6, 5; 12, 11, 10, 9; 16, 15, 14, 13. There can be no increasing subsequence of five terms, because any increasing subsequence can have only one element from each of the four groups. There can be no decreasing subsequence of five terms, because any decreasing subsequence cannot have elements from more than one group.
\end{enumerate}

\section*{\textbf{6.3 Permutations and Combinations}}
\begin{enumerate}[label=\textbf{\arabic*.}]
	\item List all the permutations of $\{ a, b, c\}$.
	
	$a, b, c;\ a, c, b;\ b, a, c;\ b, c, a;\ c, a, b; \text{ and } c, b, a$
	
	\item How many different permutations are there of the set $\{ a, b, c, d, e, f, g \}$?
	
	$P(7, 7) = \frac{7!}{(7 - 7)!} = 7! = 5040$ different permutations.
	
	\item How many permutations of $\{ a, b, c, d, e, f, g \}$ end with $a$?
	
	If we want the permutation to end with $a$, then we may as well forget about the $a$, and just count the number of permutations of $\{ b, c, d, e, f, g \}$. Therefore the answer is $P(6, 6) = 6! = 720$.
	
	\item Find the value of each of these quantities.
	
	\begin{enumerate}[label=\textbf{\alph*)}]
		\item $P(6, 3) = \frac{6!}{(6 - 3)!} = 6 \cdot 5 \cdot 4 = 120$
		\item $P(6, 5) = \frac{6!}{(6 - 5)!} = 6! = 720$
		\item $P(8, 1) = \frac{8!}{(8 - 1)!} = 8$
		\item $P(8, 5) = \frac{8!}{(8 - 5)!} = 8 \cdot 7 \cdot 6 \cdot 5 \cdot 4 = 6720$
		\item $P(8, 8) = \frac{8!}{(8 - 8)!} = 8! = 40320$
		\item $P(10, 9) = \frac{10!}{(10 - 9)!} = 10! = 3628800$
	\end{enumerate}

	\item Find the number of 5-permutations of a set with nine elements.
	
	$P(9, 5) = \frac{9!}{(9 - 5)!} = 9 \cdot 8 \cdot 7 \cdot 6 \cdot 5 = 15120$
	
	\item How many possibilities are there for the win, place, and show (first, second, and third) positions in a horse race with 12 horses if all orders of finish are possible?
	
	$P(12, 3) = \frac{12!}{(12 - 3)!} = 12 \cdot 11 \cdot 10 = 1320$
	
	\item There are six candidates for governor of a state. In how many different orders can the names of the candidates be printed on a ballot?
	
	$P(6, 6) = \frac{6!}{(6 - 6)!} = 6! = 720$
	
	\item How many bit strings of length 10 contain
	
	\begin{enumerate}[label=\textbf{\alph*)}]
		\item exactly four 1s?
		
		$C(10, 4) = \frac{10!}{4!(10 - 4)!} = 210$
		
		\item at most four 1s?
		
		$C(10, 4) + C(10, 3) + C(10, 2) + C(10, 1) + C(10, 0) = 210 + 120 + 45 + 10 + 1 = 386$
		
		\item at least four 1s?
		
		$C(10, 4) + C(10, 5) + \ldots + C(10, 10) = 210 + 252 + 210 + 120 + 45 + 10 + 1 = 848$
		
		\item an equal number of 0s and 1s?
		
		To have an equal number of 0s and 1s in this case means to have five 1s. Therefore the answer is $C(10, 5) = 252$.
	\end{enumerate}

	\item A group contains $n$ men and $n$ women. How many ways are there to arrange these people in a row if the men and women alternate?
	
	We assume that a row has a distinguished head. Consider the order in which the men appear relative to each other. There are $n$ men, and all of the $P(n, n) = n!$ arrangements is allowed. Similarly, there are $n!$ arrangements in which the women can appear. Now the men and women must alternate, and there are the same number of men and women; therefore there are exactly two possibilities: either the row starts with a man and ends with a woman ($MWMW \ldots MW$) or else it starts with a woman and ends with a man ($WMWM \ldots WM$). We have three tasks to perform, then: arrange the men among themselves, arrange the women among themselves, and decide which sex starts the row. By the product rule there $n! \cdot n! \cdot 2 = 2(n!)^2$ ways in which this can be done.
	
	\item In how many ways can a set of five letters be selected from the English alphabet?
	
	We assume that a combination is called for, not a permutation, since we are told to \emph{select a set}, not \emph{form an arrangement}. We need to choose 5 things from 26, so there are $C(26, 5) = \frac{26!}{5!(26 - 5)!} = 65780$ ways to do so.
	
	\item How many subsets with more than two elements does a set with 100 elements have?
	
	We know that there are $2^{100}$ subsets of a set with 100 elements. All of them have more than two elements except the empty set, the 100 subsets consisting of one element each, and the $C(100, 2) = 4950$ subsets with two elements. Therefore the answer is $2^{100} - 5051 \approx 1.3 \times 10^{30}$.
	
	\item A coin is flipped 10 times where each flip comes up either heads or tails. How many possible outcomes
	
	\begin{enumerate}[label=\textbf{\alph*)}]
		\item are there in total?
		
		Each flip can be either heads or tails, so there are $2^{10} = 1024$ possible outcomes. 
		
		\item contain exactly two heads?
		
		To specify an outcome that has exactly two heads, we simply need to choose the two flips that came up heads. There are $C(10, 2) = 45$ such outcomes.
		
		\item contain at most three tails?
		
		To contain at most three tails means to contain three tails, two tails, one tail, or no tails. Reasoning as in part (b), we see that there are $C(10, 3) + C(10, 2) + C(10, 1) + C(10, 0) = 120 + 45 + 10 + 1 = 176$ such outcomes.
		
		\item contain the same number of heads and tails?
		
		To have an equal number of heads and tails in this case means to have five heads. Therefore the answer is $C(10, 5) = 252$.
	\end{enumerate}

	\item How many permutations of the letters $ABCDEFG$ contain
	
	\begin{enumerate}[label=\textbf{\alph*)}]
		\item the string $BCD$?
		
		If $BCD$ is to be a substring, then we can think of that block of letters as one superletter, and the problem is to count permutations of five items---the letters $A, E, F, \text{ and } G$, and the superletter $BCD$. Therefore the answer is $P(5, 5) = 5! = 120$.
		
		\item the string $CFGA$?
		
		Reasoning as in part (a), we see that the answer is $P(4, 4) = 4! = 24$.
		
		\item the strings $BA$ and $GF$?
		
		As in part (a), we glue $BA$ into one item and glue $GF$ into one item. Therefore we need to permute five items, and there are $P(5, 5) = 5! = 120$ ways to do it.
		
		\item the strings $ABC$ and $DE$?
		
		This is similar to part (c). Glue $ABC$ into one item and glue $DE$ into one item, producing four items, so the answer is $P(4, 4) = 4! = 24$.
		
		\item the strings $ABC$ and $CDE$?
		
		If both $ABC$ and $CDE$ are substrings, then $ABCDE$ ahs to be a substring. So we are really just permuting three items: $ABCDE$, $F$, and $G$. Therefore the answer is $P(3, 3) = 3! = 6$.
		
		\item the strings $CBA$ and $BED$?
		
		There are no permutations with both of these substrings, since $B$ cannot be followed by both $A$ and $E$ at the same time.
	\end{enumerate}

	\item How many ways are there for eight men and five women to stand in a line so that no two women stand next to each other? [\emph{Hint:} First position the men and then consider possible positions for the women.]
	
	First position the men relative to each other. Since there are eight men, there are $P(8, 8)$ ways to do this. This creates nine slots where a woman (but not more than one woman) may stand: in front of the first man, between the first and second men, $\ldots$, between the seventh and eighth men, and behind the eighth man. We need to choose five of these positions, in order, for the first through fifth woman to occupy (order matters, because the women are distinct people). This can be done in $P(9, 5)$ ways. Therefore the answer is $P(8, 8) \cdot P(9, 5) = 8! \cdot 9! / 4! = 609638400$.
	
	\item How many ways are there for four men and five women to stand in a line so that
	
	\begin{enumerate}[label=\textbf{\alph*)}]
		\item all men stand together?
		
		One approach to this problem is to begin by considering the four men as a block and count the number of ways to arrange 6 objects: the five women and the block of four men. There are $P(6, 6) = 6!$ ways to arrange the women and the block of men. Then we choose the specific order of the 4 men within their block, which can be done in $P(4, 4) = 4!$ ways. And so the total number of arrangements of the nine people with the men all standing together is $6! \cdot 4! = 17280$.
		
		\item all women stand together?
		
		This is similar to (a). We begin by arranging the four men and the women considered as a block, which can be done in $5!$ ways. Then the women are arranged within their block in $5!$ ways, since there are 5 women. So the number of arrangements is $5! \cdot 5! = 14400$.
	\end{enumerate}

	\item The English alphabet contains 21 consonants and five vowels. How many strings of six lowercase letters of the English alphabet contain
	
	\begin{enumerate}[label=\textbf{\alph*)}]
		\item exactly one vowel?
		
		We need to choose the position for the vowel, and this can be done in 6 ways. Next we need to choose the vowel to use, and this can be done in 5 ways. Each of the other five positions in the string can contain any of the 21 consonants, so there are $21^5$ ways to fill the rest of the string. Therefore the answer is $6 \cdot 5 \cdot 21^5 = 122523030$.
		
		\item exactly two vowels?
		
		We need to choose the position for the vowels, and this can be done in $C(6, 2) = 15$ ways (we need to choose two positions out of six). We need to choose the two vowels ($5^2$ ways). Each of the other four positions in the string can contain any of the 21 consonants, so there are $21^4$ ways to fill the rest of the string. Therefore the answer is $15 \cdot 5^2 \cdot 21^4 = 72930375$.
		
		\item at least one vowel?
		
		The best way to do this is to count the number of strings with no vowels and subtract this from the total number of strings. We obtain $26^6 - 21^6 = 223149655$.
		
		\item at least two vowels?
		
		As in part (c), we will do this by subtracting from the total number of strings, the number of strings with no vowels and the number of strings with one vowel (this latter quantity having been computed in part (a)). We obtain $26^6 - 21^6 - 6 \cdot 5 \cdot 21^5 = 223149655 - 122523030 = 100626625$.
	\end{enumerate}
\end{enumerate}
\end{document}