\documentclass[11pt]{article}

\usepackage{color, enumitem, graphicx, amsmath, amsthm, amssymb}
\usepackage[margin=.5in]{geometry}
\usepackage[T1]{fontenc} % Use 8-bit encoding that has 256 glyphs

\usepackage[english]{babel} % English language/hyphenation

\usepackage{sectsty} % Allows customizing section commands
\allsectionsfont{\normalfont\scshape} % Make all sections centered, the default font and small caps

\usepackage{fancyhdr} % Custom headers and footers
\pagestyle{fancyplain} % Makes all pages in the document conform to the custom headers and footers
\fancyhead{} % No page header - if you want one, create it in the same way as the footers below
\fancyfoot[L]{} % Empty left footer
\fancyfoot[C]{} % Empty center footer
\fancyfoot[R]{\thepage} % Page numbering for right footer
\renewcommand{\headrulewidth}{0pt} % Remove header underlines
\renewcommand{\footrulewidth}{0pt} % Remove footer underlines
\setlength{\headheight}{13.6pt} % Customize the height of the header

\setlength\parindent{2em}

\graphicspath{ {./} }

\renewcommand\qedsymbol{$\blacksquare$}

%-------------------------------------------------------------------------------
%	TITLE SECTION
%-------------------------------------------------------------------------------

%\newcommand{\horrule}[1]{\rule{\linewidth}{#1}} % Create horizontal rule command with 1 argument of height

\title{	
	\normalfont \normalsize 
	\textsc{Discrete Mathematics} \\
	%\horrule{0.5pt} \\[0.4cm] % Thin top horizontal rule
	\huge Basic Structures: Sets, Functions, Sequences, Sums, and Matrices \\
	%\horrule{2pt} \\[0.5cm] % Thick bottom horizontal rule
}

\author{Myles Moylan} % Your name

\date{} % Today's date or a custom date


%-------------------------------------------------------------------------------
%	WORK SECTION
%-------------------------------------------------------------------------------

\begin{document}
	
\maketitle

\section*{\textbf{2.1 Sets}}
\begin{enumerate}[label=\textbf{\arabic*.}]
	\item List the members of these sets.
	
	\begin{enumerate}[label=\textbf{\alph*)}]
		\item $\{\,x \mid x \text{ is a real number such that } x^2 = 1\,\}$
		
		$\{1, -1\}$
		
		\item $\{\,x \mid x \text{ is a positive integer less than } 12\,\}$
		
		$\{1, 2, 3, 4, 5, 6, 7, 8, 9, 10, 11\}$
		
		\item $\{\,x \mid x \text{ is the square of an integer and } x < 100\,\}$
		
		$\{0, 1, 4, 9, 16, 25, 36, 49, 84, 81\}$
		
		\item $\{\,x \mid x \text{ is an integer such that } x^2 = 1\,\}$
		
		$\emptyset$ ($\sqrt{2}$ is not an integer)
	\end{enumerate}

	\item Use set builder notation to give a description of each of these sets.
	
	\begin{enumerate}[label=\textbf{\alph*)}]
		\item $\{0, 3, 6, 9, 12\}$
		
		$\{\,x \in \mathbb{N} \mid 3x \text{ and } x \leq 4\,\}$
		
		\item $\{-3, -2, -1, 0, 1, 2, 3\}$
		
		$\{\,x \in \mathbb{Z} \mid -3 \leq x \leq 3\,\}$
		
		\item $\{m, n, o, p\}$
		
		$\{\,x \in \text{English alphabet} \mid  ...\,\}$
	\end{enumerate}

	\item Which of the intervals $(0, 5)$, $(0, 5]$, $[0, 5)$, $[0, 5]$, $(1, 4]$, $[2, 3]$, $(2, 3)$ contains
	
	\begin{enumerate}[label=\textbf{\alph*)}]
		\item 0?
		
		$[0, 5) \text{ and } [0, 5]$.
		
		\item 1?
		
		$(0, 5), (0, 5], [0, 5), \text{ and } [0, 5]$.
		
		\item 2?
		
		$(0, 5), (0, 5], [0, 5), [0, 5], (1, 4], \text{ and } [2, 3]$.
		
		\item 3?
		
		$(0, 5), (0, 5], [0, 5), [0, 5], (1, 4], \text{ and } [2, 3]$.
		
		\item 4?
		
		$(0, 5), (0, 5], [0, 5), [0, 5], \text{ and } (1, 4]]$.
		
		\item 5?
		
		$(0, 5] \text{ and } [0, 5]$
	\end{enumerate}

	\pagebreak
	\item For each of these intervals, list all its elements or explain why it is empty.
	
	\begin{enumerate}[label=\textbf{\alph*)}]
		\item $[a, a]$
		
		$\{a\}$
		
		\item $[a, a)$
		
		$\{a\}$
		
		\item $(a, a]$
		
		$\{a\}$
		
		\item $(a, a)$
		
		This interval is empty because it is an open interval from $a$ to itself.
		
		\item $(a, b)$, where $a > b$
		
		$\{a - 1, a - 2, \ldots, b - 1\}$
		
		\item $[a, b]$, where $a > b$
		
		$\{a, a - 1, a - 2, \ldots, b\}$
	\end{enumerate}

	\item For each of these pairs of sets, determine whether the first is a subset of the second, the second is a subset of the first, or neither is a subset of the other.
	
	\begin{enumerate}[label=\textbf{\alph*)}]
		\item the set of airline flights form New York to New Delhi, the set of nonstop airline flights from New York to New Delhi
		
		Every element in the second set falls within the set of the first, so the second set is a subset of the first. However, since with the first set there may be intermediate stops, the first set is not a subset of the second.
		
		\item the set of people who speak English, the set of people who speak Chinese
		
		Neither is a subset of the other.
		
		\item the set of flying squirrels, the set of living creatures that can fly
		
		Since flying squirrels are living creatures that can fly, the first set is a subset of the other. And since living creatures that can fly include, say, birds, the second set is not a subset of the first.
	\end{enumerate}

	\item For each of these pairs of sets, determine whether the first is a subset of the second, the second is a subset of the first, or neither is a subset of the other.
	
	\begin{enumerate}[label=\textbf{\alph*)}]
		\item the set of people who speak English, the set of people who speak English with an Australian accent
		
		The second set is a subset of the first while the first set is not a subset of the second.
		
		\item the set of fruits, the set of citrus fruits
		
		The second set is a subset of the first and the first is not a subset of the second.
		
		\item the set of students studying discrete mathematics, the set of students studying data structures
		
		Neither is a subset of the other.
	\end{enumerate}

	\item Determine whether each of these pairs of sets are equal.
	
	\begin{enumerate}[label=\textbf{\alph*)}]
		\item $\{1, 3, 3, 3, 5, 5, 5, 5, 5\}$, $\{5, 3, 1\}$
		
		Yes.
		
		\item $\{\{1\}\}$, $\{1, \{1\}\}$
		
		No.
		
		\item $\emptyset$, $\{\emptyset\}$
		
		No.
	\end{enumerate}

	\pagebreak
	\item Suppose that $A = \{2, 4, 6\}$, $B = \{2, 6\}$, $C = \{4, 6\}$, and $D = \{4, 6, 8\}$. Determine which of these sets are subsets of which other of these sets.
	
	$A$ is not a subset of any other set.
	
	$B$ is only a subset of $A$.
	
	$C$ is a subset of $A$ and $D$.
	
	$D$ is not a subset of any other set.
	
	\item For each of the following sets, determine whether 2 is an element of that set.
	
	\begin{enumerate}[label=\textbf{\alph*)}]
		\item $\{\,x \in \mathbb{R} \mid x \text{ is an integer greater than } 1\,\}$
		
		Yes, 2 is an integer greater than 1.
		
		\item $\{\,x \in \mathbb{R} \mid x \text{ is the square of an integer}\,\}$
		
		No, 2 is not the square of an integer.
		
		\item $\{2, \{2\}\}$
		
		Yes, this set contains 2 (as well as another set, coincidentally containing 2).
		
		\item $\{\{2\}, \{\{2\}\}\}$
		
		No, it does not contain 2 but only two other sets.
		
		\item $\{\{2\}, \{2, \{2\}\}\}$
		
		No, it does not contain 2 but only two other sets.
		
		\item $\{\{\{2\}\}\}$
		
		No, it only contains another set.
	\end{enumerate}

	\item For each of the sets in Exercise 9, determine whether $\{2\}$ is an element of that set.
	
		\begin{enumerate}[label=\textbf{\alph*)}]
		\item $\{\,x \in \mathbb{R} \mid x \text{ is an integer greater than } 1\,\}$
		
		No, the set containing 2 is not an element of this set.
		
		\item $\{\,x \in \mathbb{R} \mid x \text{ is the square of an integer}\,\}$
		
		No, the set containing 2 is not an element of this set.		
		
		\item $\{2, \{2\}\}$
		
		Yes, $\{2\}$ is an element of this set.
		
		\item $\{\{2\}, \{\{2\}\}\}$
		
		Yes, $\{2\}$ is an element of this set.
		
		\item $\{\{2\}, \{2, \{2\}\}\}$
		
		Yes, $\{2\}$ is an element of this set.
		
		\item $\{\{\{2\}\}\}$
		
		No, $\{2\}$ is not an element of this set, but $\{\{2\}\}$ is.
	\end{enumerate}

	\item Determine whether each of these statements is true or false.
	
	\begin{enumerate}[label=\textbf{\alph*)}]
		\item $0 \in \emptyset$
		
		False.
		
		\item $\emptyset \in \{0\}$
		
		False. The set on the right contains only 0, not the empty set.
		
		\item $\{0\} \subset \emptyset$
		
		False. The empty set has no proper subsets.
		
		\item $\emptyset \subset \{0\}$
		
		True. Every element of the empty set is, vacuously, an element of the set $\{0\}$; and the set $\{0\}$ contains an element that is not in the empty set.
		
		\item $\{0\} \in \{0\}$
		
		False. The set $\{0\}$ is not an element of the set $\{0\}$.
		
		\item $\{0\} \subset \{0\}$
		
		False. For one set to be the proper subset of another, the two sets cannot be equal.
		
		\item $\{\emptyset\} \subseteq \{\emptyset\}$
		
		True. Every set is a subset of itself.
	\end{enumerate}

	\item Determine whether these statements are true or false.
	
	\begin{enumerate}[label=\textbf{\alph*)}]
		\item $\emptyset \in \{\emptyset\}$
		
		True. The empty set is an element of the set on the right.
		
		\item $\emptyset \in \{\emptyset, \{\emptyset\}\}$
		
		True. The empty set is an element of the set on the right.
		
		\item $\{\emptyset\} \in \{\emptyset\}$
		
		False. The set with the empty set is not an element of the set on the right.
		
		\item $\{\emptyset\} \in \{\{\emptyset\}\}$
		
		True. The set with the empty set is an element of the set on the right.
		
		\item $\{\emptyset\} \subset \{\emptyset, \{\emptyset\}\}$
		
		True. The set with the empty set is a proper subset of the set on the right.
		
		\item $\{\{\emptyset\}\} \subset \{\emptyset, \{\emptyset\}\}$
		
		False. The set of the set with the empty set is not a proper subset of the set on the right.
		
		\item $\{\{\emptyset\}\} \subset \{\{\emptyset\}, \{\emptyset\}\}$
		
		False. The set of the set with the empty set is not a proper subset of the set on the right.
	\end{enumerate}

	\item Determine whether each of these statements is true or false.
	
	\begin{enumerate}[label=\textbf{\alph*)}]
		\item $x \in \{x\}$
		
		True. $x$ is an element in the set $\{x\}$.
		
		\item $\{x\} \subseteq \{x\}$
		
		True. Every set is a subset of itself.
		
		\item $\{x\} \in \{x\}$
		
		False. The set with $x$ is not an element of $\{x\}$.
		
		\item $\{x\} \in \{\{x\}\}$
		
		True. The set with $x$ is an element of the set $\{\{x\}\}$.
		
		\item $\emptyset \subseteq \{x\}$
		
		True. The empty set is a subset of every set.
		
		\item $\emptyset \in \{x\}$
		
		False. The empty set is not an element of the set $\{x\}$.
	\end{enumerate}

	\item Suppose that $A$, $B$, and $C$ are sets such that $A \subseteq B$ and $B \subseteq C$. Show that $A \subseteq C$.
	
	Let $x \in A$. Then since $A \subseteq B$, we can conclude that $x \in B$. And since $B \subseteq C$, it follows that $x \in C$. Therefore, $A \subseteq C$.
	
	\item Find two sets $A$ and $B$ such that $A \in B$ and $A \subseteq B$.
	
	$A = \{x\}$
	
	$B = \{x, \{x\}\}$
\end{enumerate}

\pagebreak
\section*{\textbf{2.2 Set Operations}}
\begin{enumerate}[label=\textbf{\arabic*.}]
	\item Let $A$ be the set of students who live within one mile of school and let $B$ be the set of students who walk to classes. Describe the students in each of these sets.
	
	\begin{enumerate}[label=\textbf{\alph*)}]
		\item $A \cap B$
		
		The set of those students who both live within one mile of school and walk to class.
		
		\item $A \cup B$
		
		The set of those students who live within one mile of school and those who walk to class (or both).
		
		\item $A - B$
		
		The set of those students who live within one mile of school.
		
		\item $B - A$
		
		The set of those students who walk to class.
	\end{enumerate}

	\item Suppose that $A$ is the set of sophomores at your school and $B$ is the set of students in discrete mathematics at your school. Express each of these sets in terms of $A$ and $B$.
	
	\begin{enumerate}[label=\textbf{\alph*)}]
		\item the set of sophomores taking discrete mathematics in your school
		
		$A \cap B$
		
		\item the set of sophomores at your school who are not taking discrete mathematics
		
		$A - B$
		
		\item the set of students at your school who either are sophomores or are taking discrete mathematics
		
		$\overline{A \cap B}$
		
		\item the set of students at your school who either are not sophomores or are not taking discrete mathematics
		
		$\overline{A \cup B}$
	\end{enumerate}

	\item Let $A = \{1, 2, 3, 4, 5\}$ and $B = \{0, 3, 6\}$. Find
	
	\begin{enumerate}[label=\textbf{\alph*)}]
		\item $A \cup B$.
		
		$\{0, 1, 2, 3, 4, 5, 6\}$
		
		\item $A \cap B$.
		
		$\{3\}$
		
		\item $A - B$.
		
		$\{1, 2, 4, 5\}$
				
		\item $B - A$.
		
		$\{0, 6\}$
	\end{enumerate}

	\item Let $A = \{a, b, c, d, e\}$ and $B = \{a, b, c, d, e, f, g, h\}$. Find
	
	\begin{enumerate}[label=\textbf{\alph*)}]
		\item $A \cup B$.
		
		$\{a, b, c, d, r, f, g, h\}$
		
		\item $A \cap B$.
		
		$\{a, b, c, d, e\}$
		
		\item $A - B$.
		
		$\emptyset$
		
		\item $B - A$.
		
		$\{f, g, h\}$
	\end{enumerate}

	\pagebreak
	\item Prove the complementation law by showing that $\overline{\overline{A}} = A$.
	
	By definition $\overline{\overline{A}}$ is the set of elements of the universal set that are not in $\overline{A}$. To not be in $\overline{A}$ means to be in $A$, making $\overline{\overline{A}}$ the same set as $A$.
	
	$\overline{\overline{A}} = \{\,x \mid \neg x \in \overline{A}\,\} = \{\,x \mid \neg\neg x \in A\,\} = \{\,x \mid x \in A\,\} = A$
	
	\item Prove the identity laws by showing that
	
	\begin{enumerate}[label=\textbf{\alph*)}]
		\item $A \cup \emptyset = A$.
		
		The union of the set $A$ and the empty set $\emptyset$ includes every element of both. Thus, since the $\emptyset$ has no elements, their union simply equates to the set $A$.
		
		$A \cup \emptyset = \{\,x \mid x \in A \lor x \in \emptyset\,\} = \{\,x \mid x \in A \lor \textbf{F}\,\} = A$
		
		\item $A \cap U = A$.
		
		The intersection of the set $A$ and the universal set $U$ will equate to the set $A$ because the set $U$ overlaps entirely with the set $A$ and $A$ is the only other set intersected.
		
		$A \cap U = \{\,x \mid x \in A \land x \in U\,\} = \{\,x \mid x \in A \land \textbf{T}\,\} = A$
	\end{enumerate}

	\item Prove the domination laws by showing that
	
	\begin{enumerate}[label=\textbf{\alph*)}]
		\item $A \cup U = U$.
		
		The union of the set $A$ and the universal set $U$ equates to the set $U$ because the union of the universal set includes all possible elements.
		
		$A \cup U = \{\,x \mid x \in A \lor x \in U\,\} = \{\,x \mid x \in A \lor \textbf{T}\,\} = \{\,x \mid \textbf{T}\,\} = U$
		
		\item $A \cap \emptyset = \emptyset$.
		
		The intersection of a set $A$ with the empty set $\emptyset$ will result in a set with zero elements because there are no common elements between the two sets that exist.
		
		$A \cap \emptyset = \{\,x \mid x \in A \land x \in \emptyset\,\} = \{\,x \mid x \in A \land \textbf{F}\,\} = \{\,x \mid \textbf{F}\,\} = \emptyset$
	\end{enumerate}

	\item Prove the idempotent laws by showing that
	
	\begin{enumerate}[label=\textbf{\alph*)}]
		\item $A \cup A = A$.
		
		The union of the set $A$ with itself includes all contained elements, thus resulting in the set $A$.
		
		$A \cup A = \{\,x \mid x \in A \lor x \in A\,\} = \{\,x \mid x \in A\,\} = A$
		
		\item $A \cap A = A$.
		
		Since the intersection of the set $A$ with itself encompasses only those elements common to $A$, it results in the set $A$.
		
		$A \cap A = \{\,x \mid x \in A \land x \in A\,\} = \{\,x \mid x \in A\,\} = A$
	\end{enumerate}

	\item Prove the complement laws by showing that
	
	\begin{enumerate}[label=\textbf{\alph*)}]
		\item $A \cup \overline{A} = U$.
		
		The set $\overline{A}$ includes all elements not in the set $A$. So when you effect a union between the set $A$ and its complement $\overline{A}$ you get all elements in the universal set $U$.
		
		$A \cup \overline{A} = \{\,x \mid x \in A \lor x \notin A\,\} = \{\,x \mid x \in U\,\} = U$
		
		\item $A \cap \overline{A} = \emptyset$.
		
		The intersection of the set $A$ with its complement $\overline{A}$ results in the empty set $\emptyset$ since they have no elements in common.
		
		$A \cap \overline{A} = \{\,x \mid x \in A \land x \notin A\,\} = \{\,x \mid x \in \emptyset\,\} = \emptyset$
	\end{enumerate}

	\pagebreak
	\item Show that
	
	\begin{enumerate}[label=\textbf{\alph*)}]
		\item $A - \emptyset = A$.
		
		The difference of the set $A$ and the empty set $\emptyset$ is the set containing those elements that are in $A$ but not in $\emptyset$. Since there are no elements in the empty set, this equates to the set $A$.
		
		$A - \emptyset = \{\,x \mid x \in A \land x \notin \emptyset\,\} = \{\,x \mid x \in A \land \textbf{T}\,\} = \{\,x \mid x \in A\,\}= A$
		
		\item $\emptyset - A = \emptyset$.
		
		The difference of the empty set $\emptyset$ and the set $A$ will exclude all elements in $A$, leaving the empty set $\emptyset$ as the result.
		
		$\emptyset - A = \{\,x \mid x \in \emptyset \land x \notin A\,\} = \{\,x \mid \textbf{F} \land x \notin A\,\} = \{\,x \mid x \in \emptyset\,\} = \emptyset$
	\end{enumerate}

	\item Let $A$ and $B$ be sets. Prove the commutative laws by showing that
	
	\begin{enumerate}[label=\textbf{\alph*)}]
		\item $A \cup B = B \cup A$.
		
		$A \cup B = \{\,x \mid x \in A \lor x \in B\,\} = \{\,x \mid x \in B \lor x \in A\,\} = B \cup A$
		
		\item $A \cap B = B \cap A$.
		
		$A \cap B = \{\,x \mid x \in A \land x \in B\,\} = \{\,x \mid x \in B \land x \in A\,\} = B \cap A$
	\end{enumerate}

	\item Prove the absorption law by showing that if $A$ and $B$ are sets, then $A \cup (A \cap B) = A$.
	
	\begin{tabular}{c | c | c | c}
		$A$ & $B$ & $A \cap B$ & $A \cup (A \cap B)$ \\
		\hline
		1 & 1 & 1 & 1 \\
		1 & 0 & 0 & 1 \\
		0 & 1 & 0 & 0 \\
		0 & 0 & 0 & 0 
	\end{tabular}

	\item Prove the absorption law by showing that if $A$ and $B$ are sets, then $A \cap (A \cup B) = A$.
	
	\begin{tabular}{c | c | c | c}
		$A$ & $B$ & $A \cup B$ & $A \cap (A \cup B)$ \\
		\hline
		1 & 1 & 1 & 1 \\
		1 & 0 & 1 & 1 \\
		0 & 1 & 1 & 0 \\
		0 & 0 & 0 & 0
	\end{tabular}

	\item Find the sets $A$ and $B$ if $A - B = \{1, 5, 7, 8\}$, $B - A = \{2, 10\}$, and $A \cap B = \{3, 6, 9\}$.
	
	Firstly, if $A \cap B = \{3, 6, 9\}$ then both $A$ and $B$ contain 3, 6, and 9. Also, $A - B = \{1, 5, 7, 8\}$ indicates that $A$ contains 1, 5, 7, and 8 while $B - A = \{2, 10\}$ indicates that $B$ contains 2 and 10. We can derive from these facts that $A = \{1, 3, 5, 6, 7, 8, 9\}$ and $B = \{2, 3, 6, 9, 10\}$.
	
	\item Prove the De Morgan law by showing that if $A$ and $B$ are sets, then $\overline{A \cup B} = \overline{A} \cap \overline{B}$
	
	\begin{enumerate}[label=\textbf{\alph*)}]
		\item by using set builder notation and logical equivalences.
		
		$\overline{A \cup B} = \{\,x \mid x \notin A \cup B\,\}$
		
		$\hspace{1.15cm} = \{\,x \mid \neg(x \in (A \cup B))\,\}$
		
		$\hspace{1.15cm} = \{\,x \mid \neg(x \in A \lor x \in B)\,\}$
		
		$\hspace{1.15cm} = \{\,x \mid \neg(x \in A) \land \neg(x \in B)\,\}$
		
		$\hspace{1.15cm} = \{\,x \mid x \notin A \land x \notin B\,\}$
		
		$\hspace{1.15cm} = \{\,x \mid x \in \overline{A} \land x \in \overline{B}\,\}$
		
		$\hspace{1.15cm} = \{\,x \mid x \in \overline{A} \cap \overline{B}\,\}$
		
		$\hspace{1.15cm} = \overline{A} \cap \overline{B}$
		
		\item using a membership table.
		
		\begin{tabular}{c | c | c | c | c | c | c}
			$A$ & $B$ & $A \cup B$ & $\overline{A \cup B}$ & $\overline{A}$ & $\overline{B}$ & $\overline{A} \cap \overline{B}$ \\
			\hline
			1 & 1 & 1 & 0 & 0 & 0 & 0 \\
			1 & 0 & 1 & 0 & 0 & 1 & 0 \\
			0 & 1 & 1 & 0 & 1 & 0 & 0 \\
			0 & 0 & 0 & 1 & 1 & 1 & 1
		\end{tabular}
	\end{enumerate}
\end{enumerate}

\section*{\textbf{2.3 Functions}}
\begin{enumerate}[label=\textbf{\arabic*.}]
	\item Why is $f$ not a function from $\mathbb{R}$ to $\mathbb{R}$ if
	
	\begin{enumerate}[label=\textbf{\alph*)}]
		\item $f(x) = 1/x$?
		
		This function is undefined for when $x = 0$, which is one of the elements in the domain.
		
		\item $f(x) = \sqrt{x}$?
		
		When $x$ is a negative number $f(x)$ is undefined (or at best a complex number).
		
		\item $f(x) = \pm \sqrt{(x^2 + 1)}$?
		
		We must have $f(x)$ defined uniquely, but here there are two values associated with every $x$.
	\end{enumerate}

	\item Determine whether $f$ is a function from $\mathbb{Z}$ to $\mathbb{R}$ if
	
	\begin{enumerate}[label=\textbf{\alph*)}]
		\item $f(n) = \pm n$.
		
		The function $f$ must be defined uniquely for any $n$, otherwise it is not a function.
		
		\item $f(n) = \sqrt{n^2 + 1}$.
		
		This function works for all integers, both positive and negative.
		
		\item $f(n) = 1/(n^2 - 4)$.
		
		This function is not defined for $n = 2$ because it resolves to $1/0$.
	\end{enumerate}

	\item Determine whether $f$ is a function from the set of all bit strings to the set of integers if
	
	\begin{enumerate}[label=\textbf{\alph*)}]
		\item $f(S)$ is the position of a 0 bit in $S$.
		
		This is not a function because there may not be a 0 bit in $S$ or there may be more than one. In either case it violates the definition of a function since $f(S)$ must have a unique value.
		
		\item $f(S)$ is the number of 1 bits in $S$.
		
		This is a function since the number of 1 bits is always a clearly defined nonnegative integer.
		
		\item $f(S)$ is the smallest integer $i$ such that the $i$th bit of $S$ is 1 and $f(0) = 0$ when $S$ is the empty string, the string with no bits.
		
		This function does not tell what to do with an empty string of all 0's. It is thus undefined and not a function.
	\end{enumerate}

	\item Find the domain and range of these functions. Note that in each case, to find the domain, determine the set of elements assigned values by the function.
	
	\begin{enumerate}[label=\textbf{\alph*)}]
		\item the function that assigns to each nonnegative integer its last digit
		
		Both the domain and the range are the set of nonnegative integers.
		
		\item the function that assigns the next largest integer to a positive integer
		
		The domain is the set of integers. The range is the set of positive integers.
		
		\item the function that assigns to a bit string the number of one bits in the string
		
		The domain is the set of bit strings. The range is the set of nonnegative integers.
		
		\item the function that assigns to a bit string the number of bits in the string
		
		The domain is the set of bit strings. The range is the set of nonnegative integers.
	\end{enumerate}

	\item Find the domain and range of these functions. Note that in each case, to find the domain, determine the set of elements assigned values by the function.
	
	\begin{enumerate}[label=\textbf{\alph*)}]
		\item the function that assigns to each bit string the number of ones in the string minus the number of zeros in the string
		
		The domain is the set of bit strings. The range is the set of all integers.
		
		\item the function that assigns to each bit string twice the number of zeros in that string
		
		The domain is the set of bit strings. The range is the set of even natural numbers.
		
		\item the function that assigns the number of bits left over when a bit string is split into bytes (which are blocks of 8 bits)
		
		The domain is the set of bit strings. The range is the interval $\{0, 1, 2, 3, 4, 5, 6, 7\}$.
		
		\item the function that assigns to each positive integer the largest perfect square not exceeding this integer
		
		The domain is the set of positive integers. The range is $\{1, 4, 9, 16, \ldots\}$.
	\end{enumerate}

	\item Find the domain and range of these functions.
	
	\begin{enumerate}[label=\textbf{\alph*)}]
		\item the function that assigns to each pair of positive integers the maximum of these two integers
		
		Both the domain and the range are the set of positive integers.
		
		\item the function that assigns to each positive integer the number of digits 0, 1, 2, 3, 4, 5, 6, 7, 8, 9 that do not appear as decimal digits of the integer
		
		The domain is the set of positive integers. The range is $\{0, 1, 2, 3, 4, 5, 6, 7, 8, 9\}$.
		
		\item the function that assigns to a bit string the number of times the block 11 appears
		
		The domain is the set of bit strings. The domain is the set of nonnegative integers.
		
		\item the function that assigns to a bit string the numerical position of the first 1 in the string and that assigns the value 0 to a bit string consisting of all 0s
		
		The domain is the set of bit strings. The range is the set of nonnegative integers.
	\end{enumerate}

	\item Find these values.
	
	\begin{enumerate}[label=\textbf{\alph*)}]
		\item $\lfloor 1.1 \rfloor = 1$
		\item $\lceil 1.1 \rceil = 2$
		\item $\lfloor -0.1 \rfloor = -1$
		\item $\lceil -0.1 \rceil = 0$
		\item $\lceil 2.99 \rceil = 3$
		\item $\lceil -2.99 \rceil = -2$
		\item $\lfloor \frac{1}{2} + \lceil \frac{1}{2} \rceil \rfloor = 1$
		\item $\lceil \lfloor \frac{1}{2} \rfloor + \lceil \frac{1}{2} \rceil + \frac{1}{2} \rceil = 2$
	\end{enumerate}

	\item Find these values.
	
	\begin{enumerate}[label=\textbf{\alph*)}]
		\item $\lceil \frac{3}{4} \rceil = 1$
		\item $\lfloor \frac{7}{8} \rfloor = 0$
		\item $\lceil -\frac{3}{4} \rceil = 0$
		\item $\lfloor -\frac{7}{8} \rfloor = -1$
		\item $\lceil 3 \rceil = 3$
		\item $\lfloor -1 \rfloor = -1$
		\item $\lfloor \frac{1}{2} + \lceil \frac{3}{2} \rceil \rfloor = 2$
		\item $\lfloor \frac{1}{2} \cdot \lfloor \frac{5}{2} \rfloor \rfloor = 1$
	\end{enumerate}

	\item Determine whether each of these functions from $\{a, b, c, d\}$ to itself is one-to-one.
	\begin{enumerate}[label=\textbf{\alph*)}]
		\item $f(a) = b, f(b) = a, f(c) = c, f(d) = d$
		
		This function is one-to-one since it never assigns the same value to two different domain elements.
		
		\item $f(a) = b, f(b) = b, f(c) = d, f(d) = c$
		
		Since both $f(a)$ and $f(b)$ assign to $b$, this is not a one-to-one function.
		
		\item $f(a) = d, f(b) = b, f(c) = c, f(d) = d$
		
		Since both $f(a)$ and $f(d)$ assign to $d$, this is not a one-to-one function.
	\end{enumerate}

	\item Which functions in Exercise 9 are onto?
	
	The function in part \textbf{(a)} is onto because the range is all of $\{a, b, c, d\}$. The other two functions are not onto.
	
	\item Determine whether each of these functions from $\mathbb{Z}$ to $\mathbb{Z}$ is one-to-one.
	
	\begin{enumerate}[label=\textbf{\alph*)}]
		\item $f(n) = n - 1$
		
		This function is one-to-one.
		
		\item $f(n) = n^2 + 1$
		
		This function is not one-to-one since $f(1) = f(-1) = 2$.
		
		\item $f(n) = n^3$
		
		This function is one-to-one.
		
		\item $f(n) = \lceil n / 2 \rceil$
		
		This function is not one-to-one since $f(1) = f(2) = 1$.
	\end{enumerate}

	\item Which functions in Exercise 11 are onto?
	
	\begin{enumerate}[label=\textbf{\alph*)}]
		\item This function is onto since every integer is simply 1 less than $n$.
		\item This function is not onto since $n^2 + 1$ is always positive.
		\item This function is not onto since, for example, 2 is not in the range (it is not the cube of any integer).
		\item This function is onto since $f(2x) = \lceil 2x / 2 \rceil = \lceil x \rceil = x \text{ for all } x \in \mathbb{Z}$
	\end{enumerate}

	\item Determine whether the function $f: \mathbb{Z} \times \mathbb{Z} \implies \mathbb{Z}$ is onto if
	
	\begin{enumerate}[label=\textbf{\alph*)}]
		\item $f(m, n) = m + n$.
		
		Given any integer $n$, we have $f(0, n) = n$, so the function is onto.
		
		\item $f(m, n) = m^2 + n^2$.
		
		Since the range contains no negative integers, this function is not onto.
		
		\item $f(m, n) = m$.
		
		Given any integer $m$, we have $f(m, 0) = m$, so the function is onto.
		
		\item $f(m, n) = |n|$.
		
		Since the range contains no negative integers, this function is not onto.
		
		\item $f(m, n) = m - n$.
		
		Given any integer $m$, we have $f(m, 0) = m - 0 = m$, so the function is onto.
	\end{enumerate}

	\item Consider these functions from the set of teachers in a school. Under what conditions is the function one-to-one if it assigns to a teacher his or her
	
	\begin{enumerate}[label=\textbf{\alph*)}]
		\item office.
		
		If there are no shared offices then this is one-to-one.
		
		\item assigned bus to chaperone in a group of buses taking students on a field trip.
		
		As long as there is only one chaperone per bus this is one-to-one.
		
		\item salary.
		
		This is most likely not one-to-one because there will probably be teachers making the same salary, otherwise this would be one-to-one.
		
		\item social security number.
		
		This is one-to-one because no two people can have the same social security number.
	\end{enumerate}

	\item Give an example of a function from $\mathbb{N}$ to $\mathbb{N}$ that is
	
	\begin{enumerate}[label=\textbf{\alph*)}]
		\item one-to-one but not onto.
		
		$f(n) = 2n$
		
		\item onto but not one-to-one.
		
		$f(n) = \lceil n / 2 \rceil$
		
		\item both onto and one-to-one (but different from the identity function).
		
		$
		f(n) = 
		\begin{cases}
			n + 1 & \text{if n is even} \\
			n - 1 & \text{if n is odd}
		\end{cases}
		$
		
		\item neither one-to-one nor onto.
		
		$f(n) = 1$
	\end{enumerate}

	\item Determine whether each of these functions is a bijection from $\mathbb{R}$ to $\mathbb{R}$.
	
	\begin{enumerate}[label=\textbf{\alph*)}]
		\item $f(x) = 2x + 1$
		
		One way to determine whether a function is a bijection is to try to construct its inverse. This function is a bijection since its inverse is the function $g(y) = (y - 1) / 2$.
		
		\item $f(x) = x^2 + 1$
		
		This function is not a bijection because its range does not include negative numbers.
		
		\item $f(x) = x^3$
		
		This function is a bijection since it has an inverse function, namely $g(y) = y^{1/3}$.
		
		\item $f(x) = (x^2 + 1) / (x^2 + 2)$
		
		This function is not a bijection since you can see that $x$ and $-x$ have the same image.
	\end{enumerate}

	\item Let $S = \{-1, 0, 2, 4, 7\}$. Find $f(S)$ if
	
	\begin{enumerate}[label=\textbf{\alph*)}]
		\item $f(x) = 1$.
		
		$f(S) = \{1, 1, 1, 1, 1\}$
		
		\item $f(x) = 2x + 1$.
		
		$f(S) = \{-1, 1, 5, 9, 15\}$
		
		\item $f(x) = \lceil x / 5 \rceil$.
		
		$f(S) = \{0, 0, 1, 1, 2\}$
		
		\item $f(x) = \lfloor (x^2 + 1) / 3 \rfloor$.
		
		$f(S) = \{0, 0, 1, 5, 16\}$
	\end{enumerate}

	\item Let $f(x) = \lfloor x^2 / 3 \rfloor$. Find $f(S)$ if
	
	\begin{enumerate}[label=\textbf{\alph*)}]
		\item $S = \{-2, -1, 0, 1, 2, 3\}$.
		
		$f(S) = \{1, 0, 0, 0, 1, 3\}$
		
		\item $S = \{0, 1, 2, 3, 4, 5\}$.
		
		$f(S) = \{0, 0, 1, 3, 5, 8\}$
		
		\item $S = \{1, 5, 7, 11\}$.
		
		$f(S) = \{0, 8, 16, 40\}$
		
		\item $S = \{2, 6, 10, 14\}$.
		
		$f(S) = \{1, 12, 33, 65\}$
	\end{enumerate}
\end{enumerate}

\section*{\textbf{2.4 Sequences and Summations}}
\begin{enumerate}[label=\textbf{\arabic*.}]
	\item Find these terms of the sequence $\{a_n\}$, where $a_n = 2 \cdot (-3)^n + 5^n$.
	
	\begin{enumerate}[label=\textbf{\alph*)}]
		\item $a_0$
		
		$a_0 = 2 \cdot (-3)^0 + 5^0 = 2 \cdot 1 + 1 = 3$
		
		\item $a_1$
		
		$a_1 = 2 \cdot (-3)^1 + 5^1 = 2 \cdot (-3) + 5 = -1$
		
		\item $a_4$
		
		$a_4 = 2 \cdot (-3)^4 + 5^4 = 2 \cdot 81 + 625 = 787$
		
		\item $a_5$
		
		$a_5 = 2 \cdot (-3)^5 + 5^5 = 2 \cdot (-243) + 3125 = 2639$
	\end{enumerate}

	\item What is the term $a_8$ of the sequence $\{a_n\}$ if $a_n$ equals
	
	\begin{enumerate}[label=\textbf{\alph*)}]
		\item $2^{n-1}$?
		
		$2^{8 - 1} = 2^7 = 128$
		
		\item $7$?
		
		$7$
		
		\item $1 + (-1)^n$?
		
		$1 + (-1)^8 = 1 + 1 = 2$
		
		\item $-(-2)^n$?
		
		$-(-2)^8 = -256$
	\end{enumerate}

	\item What are the terms $a_0$, $a_1$, $a_2$, and $a_3$ of the sequence $\{a_n\}$, where $a_n$ equals
	
	\begin{enumerate}[label=\textbf{\alph*)}]
		\item $2^n + 1$?
		
		$a_0 = 2^0 + 1 = 1$, $a_1 = 2^1 + 1 = 3$, $a_2 = 2^2 + 1 = 5$, $a_3 = 2^3 + 1 = 9$
		
		\item $(n + 1)^{n + 1}$?
		
		$a_0 = 1^1 = 1$, $a_1 = 2^2 = 4$, $a_2 = 3^3 = 27$, $a_3 = 4^4 = 256$
		
		\item $\lfloor n / 2 \rfloor$?
		
		$a_0 = \lfloor 0 / 2 \rfloor = 0$, $a_1 = \lfloor 1 / 2 \rfloor = 0$, $a_2 = \lfloor 2 / 2 \rfloor = 1$, $a_3 = \lfloor 3 / 2 \rfloor = 1$
		
		\item $\lfloor n / 2 \rfloor + \lceil n / 2 \rceil$?
		
		$a_0 = \lfloor 0 / 2 \rfloor + \lceil 0 / 2 \rceil = 0 + 0 = 0$, $a_1 = \lfloor 1 / 2 \rfloor + \lceil 1 / 2 \rceil = 0 + 1 = 1$, $a_2 = \lfloor 2 / 2 \rfloor + \lceil 2 / 2 \rceil = 1 + 1 = 2$, $a_3 = \lfloor 3 / 2 \rfloor + \lceil 3 / 2 \rceil = 1 + 2 = 3$
	\end{enumerate}

	\item What are the terms $a_0$, $a_1$, $a_2$, and $a_3$ of the sequence $\{a_n\}$, where $a_n$ equals
	
	\begin{enumerate}[label=\textbf{\alph*)}]
		\item $(-2)^n$?
		
		$a_0 = (-2)^0 = 1$, $a_1 = (-2)^1 = -2$, $a_2 = (-2)^2 = 4$, $a_3 = (-2)^3 = -8$
		
		\item $3$?
		
		$a_0 = a_1 = a_2 = a_3 = 3$
		
		\item $7 + 4^n$?
		
		$a_0 = 7 + 4^0 = 8$, $a_1 = 7 + 4^1 = 11$, $a_2 = 7 + 4^2 = 23$, $a_3 = 7 + 4^3 = 71$
		
		\item $2^n + (-2)^n$?
		
		$a_0 = 2^0 + (-2)^0 = 2$, $a_1 = 2^1 + (-2)^1 = 0$, $a_2 = 2^2 + (-2)^2 = 8$, $a_3 = 2^3 + (-2)^3 = 0$
	\end{enumerate}

	\pagebreak
	\item List the first 10 terms of each of these sequences.
	
	\begin{enumerate}[label=\textbf{\alph*)}]
		\item the sequence that begins with 2 and in which each successive term is 3 more than the preceding term
		
		$\{2, 5, 8, 11, 14, 17, 20, 23, 26, 29\}$
		
		\item the sequence that lists each positive integer three times, in increasing order
		
		$\{1, 1, 1, 2, 2, 2, 3, 3, 3, 4\}$
		
		\item the sequence that lists the odd positive integers in increasing order, listing each odd integer twice
		
		$\{1, 1, 3, 3, 5, 5, 7, 7, 9, 9\}$
		
		\item the sequence whose $n$th term is $n! - 2^n$
		
		$\{-1, -2, -2, 8, 88, 656, 4912, 40064, 362368, 3627776\}$
		
		\item the sequence that begins with 3, where each succeeding term is twice the preceding term
		
		$\{3, 6, 12, 24, 48, 96, 192, 384, 768, 1536\}$
		
		\item the sequence whose first term is 2, second term is 4, and each succeeding term is the sum of the two preceding terms
		
		$\{2, 4, 6, 10, 16, 26, 42, 68, 110, 178\}$
		
		\item the sequence whose $n$th term is the number of bits in the binary expansion of the number $n$
		
		$\{1, 2, 2, 3, 3, 3, 3, 4, 4, 4\}$
		
		\item the sequence where the $n$th term is the number of letters in the English word for the index $n$
		
		$\{3, 3, 5, 4, 4, 3, 5, 5, 4, 3\}$
	\end{enumerate}

	\item Find the first five terms of the sequence defined by each of these recurrence relations and initial conditions.
	
	\begin{enumerate}[label=\textbf{\alph*)}]
		\item $a_n = 6a_{n - 1}$, $a_0 = 2$
		
		$a_0 = 2$ \\
		$a_1 = 6a_0 = 6 \cdot 2 = 12$ \\
		$a_2 = 6a_1 = 6 \cdot 12 = 72$ \\
		$a_3 = 6a_2 = 6 \cdot 72 = 432$ \\
		$a_4 = 6a_3 = 6 \cdot 432 = 2592$
		
		\item $a_n = a_{n - 1}^2$, $a_1 = 2$
		
		$a_1 = 2$ \\
		$a_2 = a_1^2 = 2^2 = 4$ \\
		$a_3 = a_2^2 = 4^2 = 16$ \\
		$a_4 = a_3^2 = 16^2 = 256$ \\
		$a_5 = a_4^2 = 256^2 = 65536$
		
		\item $a_n = a_{n - 1} + 3a_{n - 2}$, $a_0 = 1$, $a_1 = 2$
		
		$a_0 = 1$ \\
		$a_1 = 2$ \\
		$a_2 = a_1 + 3a_0 = 2 + 3 \cdot 1 = 5$ \\
		$a_3 = a_2 + 3a_1 = 5 + 3 \cdot 2 = 10$ \\
		$a_4 = a_3 + 3a_2 = 10 + 3 \cdot 5 = 25$
		
		\item $a_n = na_{n - 1} + n^2a_{n - 2}$, $a_0 = 1$, $a_1 = 1$
		
		$a_0 = 1$ \\
		$a_1 = 1$ \\
		$a_2 = 2a_1 + 2^2a_0 = 2 \cdot 1 + 2^2 \cdot 1 = 6$ \\
		$a_3 = 3a_2 + 3^2a_1 = 3 \cdot 6 + 3^2 \cdot 1 = 27$ \\
		$a_4 = 4a_3 + 4^2a_2 = 4 \cdot 27 + 4^2 \cdot 6 = 204$
		
		\item $a_n = a_{n - 1} + a_{n - 3}$, $a_0 = 1$, $a_1 = 2$, $a_2 = 0$
		
		$a_0 = 1$ \\
		$a_1 = 2$ \\
		$a_2 = 0$ \\
		$a_3 = a_2 + a_0 = 0 + 1 = 1$ \\
		$a_4 = a_3 + a_1 = 1 + 2 = 3$
	\end{enumerate}

	\item Find the first six terms of the sequence defined by each of these recurrence relations and initial conditions.
	
	\begin{enumerate}[label=\textbf{\alph*)}]
		\item $a_n = -2a_{n - 1}$, $a_0 = -1$
		
		$a_0 = -1$ \\
		$a_1 = -2a_0 = -2 \cdot -1 = 2$ \\
		$a_2 = -2a_1 = -2 \cdot 2 = -4$ \\
		$a_3 = -2a_2 = -2 \cdot -4 = 8$ \\
		$a_4 = -2a_3 = -2 \cdot 8 = -16$ \\
		$a_5 = -2a_4 = -2 \cdot -16 = 32$
		
		\item $a_n = a_{n - 1} - a_{n - 2}$, $a_0 = 2$, $a_1 = -1$
		
		$a_0 = 2$ \\
		$a_1 = -1$ \\
		$a_2 = a_1 - a_0 = -1 - 2 = -3$ \\
		$a_3 = a_2 - a_1 = -3 - (-1) = -2$ \\
		$a_4 = a_3 - a_2 = -2 - (-3) = 1$ \\
		$a_5 = a_4 - a_3 = 1 - (-2) = 3$
		
		\item $a_n = 3a_{n - 1}^2$, $a_0 = 1$
		
		$a_0 = 1$ \\
		$a_1 = 3a_0^2 = 3 \cdot 1^2 = 3$ \\
		$a_2 = 3a_1^2 = 3 \cdot 3^2 = 27$ \\
		$a_3 = 3a_2^2 = 3 \cdot 27 ^2 = 2187$ \\
		$a_4 = 3a_3^2 = 3 \cdot 2187^2 = 14348907$ \\
		$a_5 = 3a_4^2 = 3 \cdot 14348907^2 = 6.176733963$e$14$ \\
		
		\item $a_n = na_{n - 1} + a_{n - 2}^2$, $a_0 = -1$, $a_1 = 0$
		
		$a_0 = -1$ \\
		$a_1 = 0$ \\
		$a_2 = 2a_1 + a_0^2 = 2 \cdot 0 + (-1)^2 = 1$ \\
		$a_3 = 3a_2 + a_1^2 = 3 \cdot 1 + 0^2 = 3$ \\
		$a_4 = 4a_3 + a_2^2 = 4 \cdot 3 + 1^2 = 13$ \\
		$a_5 = 5a_4 + a_3^2 = 5 \cdot 13 + 3^2 = 74$
		
		\item $a_n = a_{n - 1} - a_{n - 2} + a_{n - 3}$, $a_0 = 1$, $a_1 = 1$, $a_2 = 2$
		
		$a_0 = 1$ \\
		$a_1 = 1$ \\
		$a_2 = 2$ \\
		$a_3 = a_2 - a_1 + a_0 = 2 - 1 + 1 = 2$ \\
		$a_4 = a_3 - a_2 + a_1 = 2 - 2 + 1 = 1$ \\
		$a_5 = a_4 - a_3 + a_2 = 1 - 2 + 1 = 0$
	\end{enumerate}

	\item Let $a_n = 2^n + 5 \cdot 3^n$ for $n = 0, 1, 2, \ldots$ .
	
	\begin{enumerate}[label=\textbf{\alph*)}]
		\item Find $a_0$, $a_1$, $a_2$, $a_3$, and $a_4$.
		
		$a_0 = 2^0 + 5 \cdot 2^0 = 6$ \\
		$a_1 = 2^1 + 5 \cdot 3^1 = 17$ \\
		$a_2 = 2^2 + 5 \cdot 3^2 = 49$ \\
		$a_3 = 2^3 + 5 \cdot 3^3 = 143$ \\
		$a_4 = 2^4 + 5 \cdot 3^4 = 421$
		
		\pagebreak
		\item Show that $a_2 = 5a_1 - 6a_0$, $a_3 = 5a_2 - 6a_1$, and $a_4 = 5a^3 - 6a_2$.
		
		$a_2 = 5 \cdot 17 - 6 \cdot 6 = 49$ \\
		$a_3 = 5 \cdot 49 - 6 \cdot 17 = 143$ \\
		$a_4 = 5 \cdot 143 - 6 \cdot 49 = 421$
		
		\item Show that $a_n = 5a_{n - 1} - 6a_{n - 2}$ for all integers $n$ with $n \geq 2$.
		
		$5a_{n - 1} - 6a_{n - 2} = 5(2^{n - 1} + 5 \cdot 3^{n - 1}) - 6(2^{n - 2} + 5 \cdot 3^{n - 2})$

		$\hspace{2.53cm} = 5 \cdot 2^{n - 1} + 25 \cdot 3^{n - 1} - 6 \cdot 2^{n - 2} - 30 \cdot 3^{n - 2}$
		
		$\hspace{2.53cm} = 5 \cdot 2^{n - 1} - 6 \cdot 2^{n - 2} + 25 \cdot 3^{n - 1} - 30 \cdot 3^{n - 2}$
		
		$\hspace{2.53cm} = 10 \cdot 2^{n - 2} - 6 \cdot 2^{n - 2} + 75 \cdot 3^{n - 2} - 30 \cdot 3^{n - 2}$
		
		$\hspace{2.53cm} = 2^{n - 2}(10 - 6) + 3^{n - 2}(75 - 30)$
		
		$\hspace{2.53cm} = 2^{n - 2} \cdot 4 + 3^{n - 2} \cdot 9 \cdot 5$
		
		$\hspace{2.53cm} = 2^n + 3^n \cdot 5 = a_n$
	\end{enumerate}

	\item Show that the sequence $\{a_n\}$ is a solution of the recurrence relation $a_n = a_{n - 1} + 2a_{n - 2} + 2n - 9$ if
	
	\begin{enumerate}[label=\textbf{\alph*)}]
		\item $a_n = -n + 2$.
		
		$a_{n - 1} + 2a_{n - 2} + 2n - 9 = -(n - 1) + 2 + 2(-(n - 2) + 2) + 2n - 9$
		
		$\hspace{3.9cm} = -n + 2 = a_n$
		
		\item $a_n = 5(-1)^n - n + 2$.
		
		$a_{n - 1} + 2a_{n - 2} + 2n - 9 = 5(-1)^{n - 1} - n + 2 + 2(5(-1)^{n - 2} - n + 2) + 2n - 9$
		
		$\hspace{3.9cm} = 5(-1)^{n - 2}(-1 + 2) - n + 2 = a_n$
		
		Note that we had to factor our $(-1)^{n - 2}$ and that this is the same as $(-1)^n$ since $(-1)^2 = 1$.
		
		\item $a_n = 3(-1)^n + 2^n - n + 2$.
		
		$a_{n - 1} + 2a_{n - 2} + 2n - 9 = 3(-1)^{n - 1} + 2^{n - 1} - (n - 1) + 2 + 2(3(-1)^{n - 2} + 2^{n - 2} - (n - 2) + 2) + 2n - 9$
		
		$\hspace{3.9cm} = 3(-1)^{n - 2}(-1 + 2) + 2^{n - 2}(2 + 2) - n + 2 = a_n$
		
		Note that we had to factor out $2^{n - 2}$ and that $2^{n - 2} \cdot 4 = 2^n$.
		
		\item $a_n = 7 \cdot 2^n - n + 2$.
		
		$a_{n - 1} + 2a_{n - 2} + 2n - 9 = 7 \cdot 2^{n - 1} - (n - 1) + 2 + 2(7 \cdot 2^{n - 2} - (n - 2) + 2) + 2n - 9$
		
		$\hspace{3.9cm} = 2^{n - 2}(7 \cdot 2 + 2 \cdot 7) - n + 2 = a_n$
	\end{enumerate}

	\item Find the solution to each of these recurrence relations and initial conditions. Use an iterative approach.
	
	\begin{enumerate}[label=\textbf{\alph*)}]
		\item $a_n = 3a_{n - 1}$, $a_0 = 2$
		
		$a_n = 3a_{n - 1}$
		
		$\hspace{0.5cm} = 3(3a_{n - 2}) = 3^2a_{n - 2}$
		
		$\hspace{0.5cm} = 3^2(3a_{n - 3}) = 3^3a_{n - 3}$
		
		$\hspace{1cm} \vdots$
		
		$\hspace{0.5cm} = 3^na_{n - n} = 3^na_0 = 3^n \cdot 2$
		
		\item $a_n = a_{n - 1} + 2$, $a_0 = 3$
		
		$a_n = a_{n - 1} + 2$
		
		$\hspace{0.5cm} = (a_{n - 2} + 2) + 2 = a_{n - 2} + (2 \cdot 2)$
		
		$\hspace{0.5cm} =  (a_{n - 3} + 2) + (2 \cdot 2) = a_{n - 3} + (3 \cdot 2)$

		$\hspace{1cm} \vdots$
		
		$\hspace{0.5cm} = a_{n - n} + (n \cdot 2) = a_0 + (n \cdot 2) = 3 + (n \cdot 2) = 3 + 2n$
		
		\item $a_n = a_{n - 1} + n$, $a_0 = 1$
		
		$a_n = n + a_{n - 1}$
		
		$\hspace{0.5cm} = n + ((n - 1) + a_{n - 2}) = (n + (n - 1)) + a_{n - 2}$
		
		$\hspace{0.5cm} = (n + (n - 1)) + ((n - 2) + a_{n - 3}) = (n + (n - 1) + (n - 2)) + a_{n - 3}$
		
		$\hspace{1cm} \vdots$
		
		$\hspace{0.5cm} = (n + (n - 1) + (n - 2) + \cdots + (n - (n - 1))) + a_{n - n}$
		
		$\hspace{0.5cm} = (n + (n - 1) + (n - 2) + \cdots + 1) + a_0$
		
		$\hspace{0.5cm} = \frac{n(n + 1)}{2} + 1 = \frac{n^2 + n + 2}{2}$
		
		\item $a_n = a_{n - 1} + 2n + 3$, $a_0 = 4$
		
		$a_n = 3 + 2n + a_{n - 1}$
		
		$\hspace{0.5cm} = 3 + 2n + (3 + 2(n - 1) + a_{n - 2}) = (2 \cdot 3 + 2n + 2(n - 1)) + a_{n - 2}$
		
		$\hspace{0.5cm} = (2 \cdot 3 + 2n + 2(n - 1)) + (3 + 2(n - 2) + a_{n - 3})$
		
		$\hspace{0.5cm} = (3 \cdot 3 + 2n + 2(n - 1) + 2(n - 2)) + a_{n - 3}$
		
		$\hspace{1cm} \vdots$
		
		$\hspace{0.5cm} = (n \cdot 3 + 2n + 2(n - 1) + 2(n - 2) + \cdots + 2(n - (n - 1))) + a_{n - n}$
		
		$\hspace{0.5cm} = (n \cdot 3 + 2n + 2(n - 1) + 2(n - 2) + \cdots + 2 \cdot 1) + a_0$
		
		$\hspace{0.5cm} = 3n +2 \cdot \frac{n(n + 1)}{2} + 4 = n^2 + 4n + 4$
		
		\item $a_n = 2a_{n - 1} - 1$, $a_0 = 1$
		
		$a_n = -1 + 2a_{n - 1}$
		
		$\hspace{0.5cm} = -1 + 2(-1 + 2a_{n - 2}) = -3 + 4a_{n - 2}$
		
		$\hspace{0.5cm} = -3 + 4(-1 + 2a_{n - 3}) = -7 + 8a_{n - 3}$
		
		$\hspace{0.5cm} = -7 + 8(-1 + 2a_{n - 4}) = -15 + 16a_{n - 4}$
		
		$\hspace{0.5cm} = -15 + 16(-1 + 2a_{n - 5}) = -31 + 32a_{n - 5}$
		
		$\hspace{1cm} \vdots$
		
		$\hspace{0.5cm} = -(2^n - 1) + 2^na_{n - n} = -2^n + 1 +2^n \cdot 1 = 1$
		
		\item $a_n = 3a_{n - 1} + 1$, $a_0 = 1$
		
		$a_n = 1 + 3a_{n - 1}$
		
		$\hspace{0.5cm} = 1 + 3(1 + 3a_{n - 2}) = (1 + 3) + 3^2a_{n - 2}$
		
		$\hspace{0.5cm} = (1 + 3) + 3^2(1 + 3a_{n - 3}) = (1 + 3 + 3^2) + 3^3a_{n - 3}$
		
		$\hspace{1cm} \vdots$
		
		$\hspace{0.5cm} = (1 + 3 + 3^2 + \cdots + 3^{n - 1}) + 3^na_{n - n}$
		
		$\hspace{0.5cm} = 1 + 3 + 3^2 + \cdots + 3^{n - 1} + 3^n$
		
		$\hspace{0.5cm} = \frac{3^{n + 1} - 1}{3 - 1}$ (a geometric series)
		
		$\hspace{0.5cm} = \frac{3^{n + 1} - 1}{2}$
		
		\item $a_n = na_{n - 1}$, $a_0 = 5$
		
		$a_n = na_{n - 1} = n(n - 1)a_{n - 2}$
		
		$\hspace{0.5cm} = n(n - 1)(n - 2)a_{n - 3} = n(n - 1)(n - 2)(n - 3)a_{n - 4}$
		
		$\hspace{1cm} \vdots$
		
		$\hspace{0.5cm} = n(n - 1)(n - 2)(n - 3) \cdots (n - (n - 1))a_{n - n}$
		
		$\hspace{0.5cm} = n(n - 1)(n - 2)(n - 3) \cdots 1 \cdot a_0$
		
		$\hspace{0.5cm} = n! \cdot 5 = 5n!$
		
		\item $a_n = 2na_{n - 1}$, $a_0 = 1$
		
		$a_n = 2na_{n - 1}$
		
		$\hspace{0.5cm} = 2n(2(n - 1)a_{n - 1}) = 2^2(n(n - 1))a_{n -2}$
		
		$\hspace{0.5cm} = 2^2(n(n - 1))(2(n - 2)a_{n - 3}) = 2^3(n(n - 1)(n - 2))a_{n - 3}$
		
		$\hspace{1cm} \vdots$
		
		$\hspace{0.5cm} = 2^nn(n - 1)(n - 2)(n - 3) \cdots (n - (n - 1))a_{n - n}$
		
		$\hspace{0.5cm} = 2^nn(n - 1)(n - 2)(n - 3) \cdots 1 \cdot a_0 = 2^nn!$
	\end{enumerate}

	\item What the values of these sums?
	
	\begin{enumerate}[label=\textbf{\alph*)}]
		\item $\sum_{k = 1}^{5} (k + 1)$
		
		$2 + 3 + 4 + 5 + 6 = 20$
		
		\item $\sum_{j = 0}^{4} (-2)^j$
		
		$1 - 2 + 4 - 8 + 16 = 11$
		
		\item $\sum_{i = 1}^{10} 3$
		
		$3 + 3 + \cdots + 3 = 10 \cdot 3 = 30$
		
		\item $\sum_{j = 0}^{8} (2^{j + 1} - 2^j)$
		
		$(4 - 2) + (8 - 4) + \cdots + (512 - 256) = -1 + 512 = 511$
	\end{enumerate}

	\item What are the values of these sums, where $S = \{1, 3, 5, 7\}$?
	
	\begin{enumerate}[label=\textbf{\alph*)}]
		\item $\sum_{j \in S} j$
		
		$1 + 3 + 5 + 7 = 16$
		
		\item $\sum_{j \in S} j^2$
		
		$1 + 9 + 25 + 49 = 84$
		
		\item $\sum_{j \in S} (-3)^j$
		
		$(-3) + (-27) + (-243) + (-2187) = -2460$
		
		\item $\sum_{j \in S} 1$
		
		$1 + 1 + 1 + 1 = 4$
	\end{enumerate}

	\item What is the value of each of these sums of terms of a geometric progression?
	
	We use the formula for the sum of a geometric progression: $\sum_{j = 0}^{n} ar^j = a(r^{n + 1} - 1) / (r - 1)$.
	
	\begin{enumerate}[label=\textbf{\alph*)}]
		\item $\sum_{j = 0}^{8} 3 \cdot 2^j$
		
		Here $a = 3$, $r = 2$, and $n = 8$, so the sum is $3(2^9 - 1) / (2 - 1) = 1533$.
		
		\item $\sum_{j = 1}^{8} 2^j$
		
		Here $a = 1$, $r = 2$, and $n = 8$. The sum taken over all the values of $j$ from 0 to $n$ is $(2^9 - 1) / (2 - 1) = 511$. However, our sum starts at $j = 1$, so we must subtract out the term that isn't there, namely $2^0$. Hence the answer is $511 - 1 = 510$.
		
		\item $\sum_{j = 2}^{8} (-3)^j$
		
		Again we have to subtract the missing terms, so the sum is $((-3)^9 - 1) / ((-3) - 1) - (-3)^0 - (-3)^1 = 4921 - 1 - (-3) = 4923$.
		
		\item $\sum_{j = 0}^{8} 2 \cdot (-3)^j$
		
		Here $a = 2$, $r = (-3)$, and $n = 8$, so the sum is $2((-3)^9 - 1) / ((-3) - 1) = 9842$.
	\end{enumerate}

	\item Find the value of each of these sums.
	
	\begin{enumerate}[label=\textbf{\alph*)}]
		\item $\sum_{j = 0}^{8} (1 + (-1)^j)$
		
		$(1 + (-1)^0) + (1 + (-1)^1) + \cdots + (1 + (-1)^8) = 10$
		
		\item $\sum_{j = 0}^{8} (3^j - 2^j)$
		
		$(3^0 - 2^0) + (3^1 - 2^1) + \cdots + (3^8 - 2^8) = 9330$
		
		\item $\sum_{j = 0}^{8} (2 \cdot 3^j + 3 \cdot 2^j)$
		
		$(2 \cdot 3^0 + 3 \cdot 2^0) + (2 \cdot 3^1 + 3 \cdot 2^1) + \cdots + (2 \cdot 3^8 + 3 \cdot 2^8) = 21215$
		
		\item $\sum_{j = 0}^{8} (2^{j + 1} - 2^j)$
		
		$(2^1 - 2^0) + (2^2 - 2^1) + \cdots + (2^9 - 2^8) = 511$
	\end{enumerate}
\end{enumerate}

\section*{\textbf{2.5 Cardinality of Sets}}
\begin{enumerate}[label=\textbf{\arabic*.}]
	\item Determine whether each of these sets is finite, countably infinite, or uncountable. For those that are countably infinite, exhibit a one-to-one correspondence between the set of positive integers and that set.
	
	\begin{enumerate}[label=\textbf{\alph*)}]
		\item the negative integers
		
		The negative integers are countably infinite. Each negative integer can be paired with its absolute value to give the desired one-to-one correspondence: $1 \iff -1$, $2 \iff -2$, $3 \iff -3$, and so on.
		
		\item the even integers
		
		The even integers are countably infinite. We can list the set of even integers in the order $0, 2, -2, 4, -4, 6, -6, ...$, and pair them with the positive integers listed in their natural order. Thus $1 \iff 0$, $2 \iff 2$, $3 \iff -2$, $4 \iff 4$, and so on.
		
		\item the integers less than 100
		
		This set is countably infinite. A formula for a correspondence with the set of positive integers is given by $f(n) = 100 - n$, listing the elements in the order $99, 98, 97, \ldots$ .
		
		\item the real numbers between 0 and $\frac{1}{2}$
		
		Since the set of real numbers between 0 and 1 is not countable, it follows that the set of real numbers between 0 and $\frac{1}{2}$ is also not countable.
		
		\item the positive integers less than 1,000,000,000
		
		This set is finite, with a cardinality of 999,999,999.
		
		\item the integers that are multiples of 7
		
		This set is countably infinite. The correspondence is given by pairing the positive integer $n$ with $7n/2$ if $n$ is even and $-7(n - 1)/2$ if $n$ is odd: $0, 7, -7, 14, -14, 21, -21, \ldots$.
	\end{enumerate}

	\item Determine whether each of these sets is countable or uncountable. For those that are countably infinite, exhibit a one-to-one correspondence between the set of positive integers and that set.
	
	\begin{enumerate}[label=\textbf{\alph*)}]
		\item all bit strings not containing the bit 0
		
		This set is countably infinite. The correspondence matches the positive integer $n$ with the string of $n - 1$ 1's.
		
		\item all positive rational numbers that cannot be written with denominators less than 4
		
		Since this is a subset of the set of rational numbers, it is countable. To find a correspondence, we just follow the path in Example 4 of the book, but omit fractions in the top three rows.
		
		\item the real numbers not containing 0 in their decimal representations
		
		This set is uncountable since real numbers are not countable.
		
		\item the real numbers containing only a finite number of 1s in their decimal representation
		
		This set is uncountable since real numbers are not countable.
	\end{enumerate}

	\item Show that a finite group of guests arriving at Hilbert's fully occupied Grand Hotel can be given rooms without evicting any current guests.
	
	Suppose $m$ new guests arrive at the fully occupied hotel. If we move the guest in Room 1 to Room $m + 1$, the guest in Room 2 to Room $m + 2$, and so on, then rooms with numbers 1 to $m$ will be vacant and the new guests can occupy them.
	
	\item Suppose that Hilbert's Grand Hotel is fully occupied on the day the hotel expands to a second building which also contains a countably infinite number of rooms. Show that the current guests can be spread out to fill every room of the two buildings of the hotel.
	
	We can use the guests in the even-numbered rooms to occupy the original rooms, and the guests in the odd-numbered rooms to occupy the rooms in the second building. For each positive integer $n$, put the guest currently in Room $2n$ into Room $n$, and the guest currently in Room $2n - 1$ into Room $n$ of the new building.
	
	\item Give an example of two uncountable sets $A$ and $B$ such that $A \cap B$ is
	
	In each case, we can make the intersection what we want it to be, and then put additional elements into $A$ and into $B$ (with no overlap) to make them uncountable.
	
	\begin{enumerate}[label=\textbf{\alph*)}]
		\item finite.
		
		We can make $A \cap B = \emptyset$. So, for example, take $A$ to be the interval $(1, 2)$ of real numbers and take $B$ to be the interval $(3, 4)$.
		
		\item countably infinite.
		
		Adjoin the set of positive integers to the examples from part (a). Then, let $A = (1, 2) \cup \mathbb{Z^+}$ and let $B = (3, 4) \cup \mathbb{Z^+}$.		
		\item uncountable.
		
		Let $A = (1, 2)$ and $B = (3, 4)$.
	\end{enumerate}

	\item Explain why the set $A$ is countable if and only if $|A| \leq |\mathbb{Z^+}|$.
	
	Suppose that $A$ is countable. This means either that $A$ is finite or that there exists a one-to-one correspondence $f$ from $A$ to $\mathbb{Z^+}$. In the former case, there is a one-to-one function $g$ from $A$ to a subset of $\mathbb{Z^+}$ (the range of $g$ is the first $n$ positive integers, where $|A| = n$). Conversely, suppose that $|A| \leq |\mathbb{Z^+}|$. By definition, this means that there is a one-to-one function $g$ from $A$ to $\mathbb{Z^+}$, so that $A$ has the same cardinality as a subset of $\mathbb{Z^+}$ (namely, the range of $g$).
	
	\item If $A$ is an uncountable set and $B$ is a countable set, must $A - B$ be uncountable?
	
	Yes. First, note that $A = (A \cap B) \cup (A - B)$. Given that $B$ is countable, its subset $A \cap B$ is also countable. If $A - B$ were also countable, then, since the union of two countable sets is also countable, we would conclude that $A$ is countable. But we are given that $A$ is not countable. Therefore our assumption that $A - B$ is countable is wrong, concluding that it is actually uncountable.
	
	\item Show that if $A$, $B$, $C$, and $D$ are sets with $|A| = |B|$ and $|C| = |D|$, then $|A \times C| = |B \times D|$.
	
	By what we are given, we know that there are bijections $f$ from $A$ to $B$ and $g$ from $C$ to $D$. Then we can define a bijection from $A \times C$ to $B \times D$ by sending $(a, c)$ to $(f(a), g(c))$. This is one-to-one and onto, so we have shown that $A \times C$ and $B \times D$ have the same cardinality.
	
	\item Show that if $A$, $B$, and $C$ are sets such that $|A| \leq |B|$ and $|B| \leq |C|$, then $|A| \leq |C|$.
	
	The definition of $|A| \leq |B|$ is that there is a one-to-one function $f : A \implies B$. Similarly, we are given a one-to-one function $g : B \implies C$. It happens that to composition $g \circ f : A \implies C$ is one-to-one. Therefore by definition $|A| \leq |C|$.
	
	\item Show that if $A$ is an infinite set, then it contains a countably infinite subset.
	
	Define a sequence $a_1, a_2, a_3, \ldots$ of elements of $A$ as follows. First, $a_1$ is any element of $A$. Once we have selected $a_1, a_2, a_3, \ldots, a_k$, let $a_{k + 1}$ be any element of $A - \{a_1, a_2, \ldots, a_k\}$. Such an element must exist because $A$ is infinite. The resulting set $\{a_1, a_2, a_3, \ldots\}$ is the desired countably infinite subset of $A$.
	
	\item Prove that if it is possible to label each element of an infinite set $S$ with a finite string of keyboard characters, from a finite list of characters, where no two elements of $S$ have the same label, then $S$ is a countably infinite set.
	
	The set of finite strings of characters over a finite alphabet is countably infinite, because we can list these strings in alphabetical order by length. For example, if the alphabet is $\{a, b, c\}$, then our list is $\lambda, a, b, c, aa, ab, ac, ba, bb, bc, ca, cb, cc, aaa, aab, \ldots$ (where $\lambda$ denotes the empty string). Therefore the infinite set $S$ can be identified with an infinite subset of this countable set, which is also countably infinite.
	
	\item Use the Schr{\"o}der-Bernstein theorem to show that $(0, 1)$ and $[0, 1]$ have the same cardinality.
	
	It suffices to fine one-to-one function $f : (0, 1) \implies [0, 1]$ and $g : [0, 1] \implies (0, 1)$. We can use the function $f(x) = x$ in the first case. For the second, we can compress $[0, 1]$ into, say, $[\frac{1}{3}, \frac{2}{3}]$; the increasing linear function $g(x) = (x + 1) / 3$ will do that. It then follows from the Schr{\"o}der-Bernstein theorem that $|(0, 1)| = |[0, 1]|$.
\end{enumerate}

\section*{\textbf{2.6 Matrices}}
\begin{enumerate}[label=\textbf{\arabic*.}]
	\item Let $\textbf{A} =	
	\begin{bmatrix}
		1 & 1 & 1 & 3 \\
		2 & 0 & 4 & 6 \\
		1 & 1 & 3 & 7
	\end{bmatrix}$.
	
	\begin{enumerate}[label=\textbf{\alph*)}]
		\item What size is \textbf{A}?
		
		Since \textbf{A} has 3 rows and 4 columns, its size is $3 \times 4$.
		
		\item What is the third column of \textbf{A}?
		
		It is the $3 \times 1$ matrix 
		$\begin{bmatrix}
			1 \\
			4 \\
			3
		\end{bmatrix}$.
		
		\item What is the second row of \textbf{A}?
		
		It is the $1 \times 4$ matrix 
		$\begin{bmatrix}
			2 & 0 & 4 & 6
		\end{bmatrix}$.
		
		\item What is the element of \textbf{A} in the (3, 2)th position?
		
		It is the element in the third row, second column, namely 1.
		
		\item What is $\textbf{A}^t$?
		
		The transpose of \textbf{A} is the $4 \times 3$ matrix 
		$\begin{bmatrix}
			1 & 2 & 1 \\
			1 & 0 & 1 \\
			1 & 4 & 3 \\
			3 & 6 & 7 
		\end{bmatrix}$.
	\end{enumerate}

	\item Find $\textbf{A} + \textbf{B}$ where
	
	\begin{enumerate}[label=\textbf{\alph*)}]
		\item $\textbf{A} = 
		\begin{bmatrix}
			1 & 0 & 4 \\
			-1 & 2 & 2 \\
			0 & -2 & -3
		\end{bmatrix}$, $\textbf{B} =
		\begin{bmatrix}
			-1 & 3 & 5 \\
			2 & 2 & -3 \\
			2 & -3 & 0
		\end{bmatrix}$.
	
		$\begin{bmatrix}
			1 & 0 & 4 \\
			-1 & 2 & 2 \\
			0 & -2 & -3
		\end{bmatrix} +
		\begin{bmatrix}
			-1 & 3 & 5 \\
			2 & 2 & -3 \\
			2 & -3 & 0
		\end{bmatrix} = 
		\begin{bmatrix}
			0 & 3 & 9 \\
			1 & 4 & -1 \\
			2 & -5 & -3
		\end{bmatrix}$
	
		\item $\textbf{A} = 
		\begin{bmatrix}
			-1 & 0 & 5 & 6 \\
			-4 & -3 & 5 & -2
		\end{bmatrix}$, $\textbf{B} =
		\begin{bmatrix}
			-3 & 9 & -3 & 4 \\
			0 & -2 & -1 & 2
		\end{bmatrix}$.
	
		$\begin{bmatrix}
			-1 & 0 & 5 & 6 \\
			-4 & -3 & 5 & -2
		\end{bmatrix} +
		\begin{bmatrix}
			-3 & 9 & -3 & 4 \\
			0 & -2 & -1 & 2
		\end{bmatrix} =
		\begin{bmatrix}
			-4 & 9 & 2 & 10 \\
			-4 & -5 & 4 & 0
		\end{bmatrix}$
	\end{enumerate}

	\item Find \textbf{AB} if
	
	\begin{enumerate}[label=\textbf{\alph*)}]
		\item $\textbf{A} = 
		\begin{bmatrix}
			2 & 1 \\
			3 & 2
		\end{bmatrix}$, $\textbf{B} =
		\begin{bmatrix}
			0 & 4 \\
			1 & 3
		\end{bmatrix}$.
	
		$=
		\begin{bmatrix}
			2 \cdot 0 + 1 \cdot 1 & 2 \cdot 4 + 1 \cdot 3 \\
			3 \cdot 0 + 2 \cdot 1 & 3 \cdot 4 + 2 \cdot 3 
		\end{bmatrix} = 
		\begin{bmatrix}
			1 & 11 \\
			2 & 18
		\end{bmatrix}$
	
		\item $\textbf{A} = 
		\begin{bmatrix}
			1 & -1 \\
			0 & 1 \\
			2 & 3
		\end{bmatrix}$, $\textbf{B} = 
		\begin{bmatrix}
			3 & -2 & -1 \\
			1 & 0 & 2
		\end{bmatrix}$.
	
		$ =
		\begin{bmatrix}
			1 \cdot 3 + (-1) \cdot 1 & 1 \cdot (-2) + (-1) \cdot 0 & 1 \cdot (-1) + (-1) \cdot 2 \\
			0 \cdot 3 + 1 \cdot 1 & 0 \cdot (-2) + 1 \cdot 0 & 0 \cdot (-1) + 1 \cdot 2 \\
			2 \cdot 3 + 3 \cdot 1 & 2 \cdot (-2) + 3 \cdot 0 & 2 \cdot (-1) + 3 \cdot 2
		\end{bmatrix}$
	
		$ = 
		\begin{bmatrix}
			2 & -2 & -3 \\
			1 & 0 & 2 \\
			9 & -4 & 4
		\end{bmatrix}$
	
		\item $\textbf{A} =
		\begin{bmatrix}
			4 & -3 \\
			3 & -1 \\
			0 & -2 \\
			-1 & 5
		\end{bmatrix}$, $\textbf{B} = 
		\begin{bmatrix}
			-1 & 3 & 2 & -2 \\
			0 & -1 & 4 & -3
		\end{bmatrix}$.
	
		$= 
		\begin{bmatrix}
			4 \cdot (-1) + (-3) \cdot 0 & 4 \cdot 3 + (-3) \cdot (-1) & 4 \cdot 2 + (-3) \cdot 4 & 4 \cdot (-2) + (-3) \cdot (-3) \\
			3 \cdot (-1) + (-1) \cdot 0 & 3 \cdot 3 + (-1) \cdot (-1) & 3 \cdot 2 + (-1) \cdot 4 & 3 \cdot (-2) + (-1) \cdot (-3) \\
			0 \cdot (-1) + (-2) \cdot 0 & 0 \cdot 3 + (-2) \cdot (-1) & 0 \cdot 2 + (-2) \cdot 4 & 0 \cdot (-2) + (-2) \cdot (-3) \\
			(-1) \cdot (-1) + 5 \cdot 0 & (-1) \cdot 3 + 5 \cdot (-1) & (-1) \cdot 2 + 5 \cdot 4 & (-1) \cdot (-2) + 5 \cdot (-3)
		\end{bmatrix}$
	
		$ = 
		\begin{bmatrix}
			-4 & 15 & -4 & 1 \\
			-3 & 10 & 2 & -3 \\
			0 & 2 & -8 & 6 \\
			1 & -8 & 18 & -13
		\end{bmatrix}$
	\end{enumerate}

	\item Find the product \textbf{AB}, where
	
	\begin{enumerate}[label=\textbf{\alph*)}]
		\item $\textbf{A} =
		\begin{bmatrix}
			1 & 0 & 1 \\
			0 & -1 & -1 \\
			-1 & 1 & 0
		\end{bmatrix}$, $\textbf{B} = 
		\begin{bmatrix}
			0 & 1 & -1 \\
			1 & -1 & 0 \\
			-1 & 0 & 1
		\end{bmatrix}$.
	
		$= 
		\begin{bmatrix}
			1 \cdot 0 + 0 \cdot 1 + 1 \cdot (-1) & 1 \cdot 1 + 0 \cdot (-1) + 1 \cdot 0 &1 \cdot (-1) + 0 \cdot 0 + 1 \cdot 1 \\
			0 \cdot 0 + (-1) \cdot 1 + (-1) \cdot (-1) & 0 \cdot 1 + (-1) \cdot (-1) + (-1) \cdot 0 & 0 \cdot (-1) + (-1) \cdot 0 + (-1) \cdot 1\\
			(-1) \cdot 0 + 1 \cdot 1 + 0 \cdot (-1) & (-1) \cdot 1 + 1 \cdot (-1) + 0 \cdot 0 & (-1) \cdot (-1) + 1 \cdot 0 + 0 \cdot 1
		\end{bmatrix}$
	
		$ = 
		\begin{bmatrix}
			-1 & 1 & 0 \\
			0 & 1 & -1 \\
			1 & -2 &1
		\end{bmatrix}$
	
		\item $\textbf{A} =
		\begin{bmatrix}
			0 & -1 \\
			7 & 2 \\
			-4 & -3
		\end{bmatrix}$, $\textbf{B} = 
		\begin{bmatrix}
			4 & -1 & 2 & 3 & 0 \\
			-2 & 0 & 3 & 4 & 1
		\end{bmatrix}$.
	
		$= 
		\begin{bmatrix}
			0 \cdot 4 + (-1) \cdot (-2) & 0 \cdot (-1) + (-1) \cdot 0 & 0 \cdot 2 + (-1) \cdot 3 & 0 \cdot 3 + (-1) \cdot 4 & 0 \cdot 0 + (-1) \cdot 1 \\
			7 \cdot 4 + 2 \cdot (-2) & 7 \cdot (-1) + 2 \cdot 0 & 7 \cdot 2 + 2 \cdot 3 & 7 \cdot 3 + 2 \cdot 4 & 7 \cdot 0 + 2 \cdot 1 \\
			(-4) \cdot 4 + (-3) \cdot (-2) & (-4) \cdot (-1) + (-3) \cdot 0 & (-4) \cdot 2 + (-3) \cdot 3 & (-4) \cdot 3 + (-3) \cdot 4 & (-4) \cdot 0 + (-3) \cdot 1 
		\end{bmatrix}$
	
		$=
		\begin{bmatrix}
			2 & 0 & -3 & -4 & -1 \\
			24 & -7 & 20 & 31 & 2 \\
			-10 & 4 & -17 & -24 & -3
		\end{bmatrix}$
	\end{enumerate}

	\item Let \textbf{A} be an $m \times n$ matrix and let \textbf{0} be the $m \times n$ matrix that has all entries equal to zero. Show that $\textbf{A} = \textbf{0} + \textbf{A} = \textbf{A} + \textbf{0}$.
	
	Since the $(i, j)$th entry of $\textbf{0} + \textbf{A}$ is the sum of the $(i, j)$th entry of \textbf{0} (namely 0) and the $(i, j)$th entry of \textbf{A}, this entry is the same as the $(i, j)$th entry of \textbf{A}. Therefore by the definition of matrix equality, $\textbf{0} + \textbf{A} = \textbf{A}$ and $\textbf{A} + \textbf{0} = \textbf{A}$.
	
	\item Show that matrix addition is commutative; that is, show that if \textbf{A} and \textbf{B} are both $m \times n$ matrices, then $\textbf{A} + \textbf{B} = \textbf{B} + \textbf{A}$.
	
	The $(i, j)$th entry of $\textbf{A} + \textbf{B}$ is the same as the $(i, j)$th entry of $\textbf{B} + \textbf{A}$. It will result in the same sum, just the order of the numbers in the calculation will be swapped.
	
	\item Show that matrix addition is associative; that is, show that if \textbf{A}, \textbf{B}, and \textbf{C} are all $m \times n$ matrices, then $\textbf{A} + (\textbf{B} + \textbf{C}) = (\textbf{A} + \textbf{B}) + \textbf{C}$.
		
	We simply look at the $(i, j)$th entries of each side. The $(i, j)$th entry of the left-hand side is $a_{ij} + (b_{ij} + c_{ij})$. The $(i, j)$th entry of the right-hand side is $(a_{ij} + b_{ij}) + c_{ij}$. By the associativity law for real number addition, these are equal.
	
	\pagebreak
	\item What do we know about the sizes of the matrices \textbf{A} and \textbf{B} if both of the products \textbf{AB} and \textbf{BA} are defined?
	
	In order for \textbf{AB} to be defined, the number of columns of \textbf{A} must be equal to the number of rows of \textbf{B}. In order for \textbf{BA} to be defined, the number of columns of \textbf{B} must be equal to the number of rows of \textbf{A}. Thus for some positive integers $m$ and $n$, it must be the case that \textbf{A} is an $m \times n$ matrix and \textbf{B} is an $n \times m$ matrix. Another way to say this is to say that \textbf{A} must have the same size as $\textbf{B}^t$.
	
	\item Let
	
	$\textbf{A} = 
	\begin{bmatrix}
		1 & 0 & 1 \\
		1 & 1 & 0 \\
		0 & 0 & 1
	\end{bmatrix}$\quad and\quad $\textbf{B} = 
	\begin{bmatrix}
		0 & 1 & 1 \\
		1 & 0 & 1 \\
		1 & 0 & 1
	\end{bmatrix}$.

	Find
	
	\begin{enumerate}[label=\textbf{\alph*)}]
		\item $\textbf{A} \lor \textbf{B}$.
		
		$=
		\begin{bmatrix}
			1 \lor 0 & 0 \lor 1 & 1 \lor 1 \\
			1 \lor 1 & 1 \lor 0 & 0 \lor 1 \\
			0 \lor 1 & 0 \lor 0 & 1 \lor 1
		\end{bmatrix} =
		\begin{bmatrix}
			1 & 1 & 1 \\
			1 & 1 & 1 \\
			1 & 0 & 1
		\end{bmatrix}$
		
		\item $\textbf{A} \land \textbf{B}$.
		
		$=
		\begin{bmatrix}
			1 \land 0 & 0 \land 1 & 1 \land 1 \\
			1 \land 1 & 1 \land 0 & 0 \land 1 \\
			0 \land 1 & 0 \land 0 & 1 \land 1
		\end{bmatrix} =
		\begin{bmatrix}
			0 & 0 & 1 \\
			1 & 0 & 0 \\
			0 & 0 & 1
		\end{bmatrix}$
		
		\item $\textbf{A} \odot \textbf{B}$.
		
		$= 
		\begin{bmatrix}
			(1 \land 0) \lor (0 \land 1) \lor (1 \land 1) & (1 \land 1) \lor (0 \land 0) \lor (1 \land 0) & (1 \land 1) \lor (0 \land 1) \lor (1 \land 1) \\
			(1 \land 0) \lor (1 \land 1) \lor (0 \land 1) & (1 \land 1) \lor (1 \land 0) \lor (0 \land 0) & (1 \land 1) \lor (1 \land 1) \lor (0 \land 1) \\
			(0 \land 0) \lor (0 \land 1) \lor (1 \land 1) & (0 \land 1) \lor (0 \land 0) \lor (1 \land 0) & (0 \land 1) \lor (0 \land 1) \lor (1 \land 1)
		\end{bmatrix}$
	
		$=
		\begin{bmatrix}
			1 & 1 & 1 \\
			1 & 1 & 1 \\
			1 & 0 & 1
		\end{bmatrix}$
	\end{enumerate}

	\item Let
	
	$\textbf{A} = 
	\begin{bmatrix}
		1 & 0 & 0 \\
		1 & 0 & 1 \\
		0 & 1 & 0
	\end{bmatrix}$.

	Find
	
	\begin{enumerate}[label=\textbf{\alph*)}]
		\item $\textbf{A}^{[2]}$.
		
		$=
		\begin{bmatrix}
			1 & 0 & 0 \\
			1 & 1 & 0 \\
			1 & 0 & 1
		\end{bmatrix}$
		
		\item $\textbf{A}^{[3]}$.
		
		$=
		\begin{bmatrix}
			1 & 0 & 0 \\
			1 & 0 & 1 \\
			1 & 1 & 0
		\end{bmatrix}$
	
		\item $\textbf{A} \lor \textbf{A}^{[2]} \lor \textbf{A}^{[3]}$.
		
		$=
		\begin{bmatrix}
			1 & 0 & 0 \\
			1 & 1 & 1 \\
			1 & 1 & 1
		\end{bmatrix}$
	\end{enumerate}
\end{enumerate}
\end{document}